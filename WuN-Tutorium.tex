\documentclass{lecturenote}
% \usepackage[T1]{fontenc}
\usepackage[utf8]{inputenc}
\usepackage{etex}
\usepackage{paralist}
\usepackage{enumitem}
\usepackage{makeidx}
\usepackage{amsmath}
\usepackage{amssymb}
\usepackage{amsfonts}
\usepackage{empheq}
\usepackage{amsthm}
\usepackage{amscd}
\usepackage[ngerman]{babel}
% \usepackage{romannum}
\usepackage{tabularx}
\let\checkmark\relax
\usepackage{dingbat}
\usepackage[misc]{ifsym}
\usepackage{soul}% for underlining
\usepackage{cancel}% for cancellign matheqs

\usepackage{stmaryrd}
\usepackage{graphicx}
\usepackage{tikz}
\usetikzlibrary{arrows,shadows} %
\usepackage[siunitx,european]{circuitikz}
\usepackage{cancel}
\usepackage{enumitem}
\usepackage{trsym}
\usepackage[
nonumberlist, %keine Seitenzahlen anzeigen
acronym,      %ein Abkürzungsverzeichnis erstellen
toc,          %Einträge im Inhaltsverzeichnis
section]      %im Inhaltsverzeichnis auf section-Ebene erscheinen
{glossaries}
\let\acs\gls
% Index
\usepackage{makeidx}
\makeindex
\hypersetup{linkcolor=black}
\usepackage{environ}
% Theoreme/Definitionen/...
% \usepackage{ntheorem}
\theoremstyle{definition}
\newtheorem{definition}{Definition}

\newcommand{\Ohm}{\Omega}
\newcommand{\eq}{=}
\newcommand{\komma}{,}
\newcommand{\cn}[1]{\underline{#1}}
\renewcommand{\Re}[1]{\operatorname{Re}\left\{#1\right\}}
\renewcommand{\Im}[1]{\operatorname{Im}\left\{#1\right\}}
\newcommand{\real}{\mathbb{R}}
\newcommand{\complex}{\mathbb{C}}
\newcommand{\diff}{\,\mathrm{d}}
\newcommand{\LT}[1]{\mathfrak{L}\left\{#1\right\}}
\DeclareMathOperator{\grad}{grad}
\let\angle\sphericalangle

% Glossary settings
\deftranslation[to=ngerman]{Acronyms}{Abkürzungsverzeichnis}
\newglossary[slg]{symbolslist}{syi}{syg}{Symbolverzeichnis}

% Color settings
\definecolor{emphlight}{rgb}{0.8,0.85,1}
% cmds
\newcommand\hinweis[2]{%
  \noindent%
  \fcolorbox{tuWhite}{tuLightGray!50}{%
  \begin{minipage}[t]{0.05\linewidth}%
    \vspace{0pt}%
    \centering%
    #1%
  \end{minipage}%
  \hspace*{0.01\linewidth}%
  \begin{minipage}[t]{0.94\linewidth}%
    \vspace{0pt}
    \itshape%
    #2%
  \end{minipage}%
  }%
  \medskip%
}

\newcommand\klausurhinweis[1]{%
  \hinweis{\large\PaperPortrait}{#1}
}

\newcommand\infohinweis[1]{%
  \hinweis{\Large\bfseries \textcircled{\large i}}{#1}
}

% colored underline
\newsavebox\MBox
\newcommand\cul[2][red]{{\sbox\MBox{$#2$}%
  \rlap{\usebox\MBox}\color{#1}\rule[-1.2\dp\MBox]{\wd\MBox}{0.5pt}}}

\newsavebox\skillbox
\newcommand\skilltitle{Skills}
\NewEnviron{skills}{%
  \begin{tikzpicture}[every shadow/.style={fill=black!30}]
  \def\firstnode{
  \node[drop shadow,fill=gray!20] (box) {
  \begin{minipage}{\textwidth}
  \begin{list}{\textcolor{tuGreen}{\checkmark}}{}
    \BODY
  \end{list}
  \end{minipage}
  };}
  \firstnode
  \node[fill=gray!40, yshift=-1, drop shadow, anchor=south west] at (box.north west) {\skilltitle};
  \firstnode
  \end{tikzpicture}
}

\newcommand\skillref[2]{\hyperref[skill:#1]{#2}}
\newcommand\skilllabel[1]{\label{skill:#1}}

\newcommand\Index[1]{\index{#1}}
% \newcommand\dh{\mbox{d.\,h.}\xspace}

% Titlepage settings
\course{WuN-Tutorium}
\author{Lennert van derWall\\In \LaTeX\
  von Enrico Jörns und Sebastian Willenborg}
% \semester{WS 2010/2011}

\makeindex
\makeglossaries

\begin{document}

% \addtokomafont{chapter}{\LARGE}

\newglossaryentry{zi-response}{%
  name={zero input response},
  description={Antwort des Systems bei $e(t)=0$ und beliebigen Anfangswerten
    $\vec{x}$}}

\newglossaryentry{zs-response}{%
  name={zero state response},
  description={Zustand des Systems bei dem alle $\vec{x}=0$ sind}
}

\newglossaryentry{asympStabil}{%
  name={Asymptotische Stabilität},
  description={Gegeben, wenn bei unerregtem Netzwerkmodell ($e(t)=0$)
    und beliebigen freien Zuständen ($\vec{x_{t0}}$) $a(t)$ für $t\to\infty$
    gegen $0$ geht.\\
    Dies ist bei einem nach Satz 5.1 beschriebenen Netzwerk der Fall, wenn 
    alle natürlichen Frequenzen des Netzwerkmodells einen echt negativen Realteil
    haben.}
}

\newglossaryentry{natFreq}{%
  name={natürliche Frequenzen},
  description={$e^{\cn{A}_i}$, $\cn{A}_i$ natürliche Frequenzen}
}

\newglossaryentry{hurwitzpolynom}{%
  name={Hurwitzpolynom},
  description={Ein reelles Polynom,
    dessen Nullstellen alle einen echt negativen Realteil haben.}
}


\glsaddall
%Befehle für Symbole
\newcommand{\newsymbolentry}[3]{%
  \newglossaryentry{symb:#1}{%
  name=#2,%
  description={#3.},%
  sort=symbol#1, type=symbolslist%
  }
}

\newsymbolentry{nQ}{$n_Q$}{Anzahl der Quellen}
\newsymbolentry{et}{$e(t)$}{Eingangssignal}
\newsymbolentry{at}{$a(t)$}{Ausgangssignal}
\newsymbolentry{ts}{$t_s$}{Schaltzeitpunkt}
\newsymbolentry{Ai}{$\cn{A}_i$}{natürliche Frequenz}
\newsymbolentry{azit}{$a_{zi}(t)$}{\itshape "`zero input response"'}
\newsymbolentry{xts}{$\vec{x}_{t_0}$}{Anfangszustandsvektor}
\newsymbolentry{aezt}{$a_{ez}(t)$}{Antwort im eingeschwungenen Zustand}
\newsymbolentry{nA}{$n_A$}{Anzahl .. Variablen}
\newsymbolentry{nR}{$n_R$}
  {Anzahl zustandsreduzierender algebraischer Gleichungen}
\newsymbolentry{adeltat}{$a_\delta(t)$}{Sprungantwort}
\newsymbolentry{Sprung}{$\Theta(t)$}{Sprungfunktion}

\glsaddall

\newacronym{SPQ}{SPQ}{Spannungsquelle}
\newacronym{STQ}{STQ}{Stromquelle}
\newacronym{FG}{FG}{Frequenzgang}
\newacronym{UF}{ÜF}{Übertragungsfunktion}
\newacronym{NWM}{NWM}{Netzwerkmodell}
\newacronym{MGL}{MGL}{Maschengleichung}
\newacronym{SGL}{SGL}{Schnittmengengleichung}
\newacronym{KWSR}{KWSR}{Komplexe Wechselstromrechnung}
\newacronym{ZGL}{ZGL}{Zweiggleichung}
\newacronym{HE}{HE}{Halbebene}
\newacronym{PBZ}{PBZ}{Partialbruchzerlegung}
\newacronym{EZ}{EZ}{Eingeschwungener Zustand}
\newacronym{HEZ}{HEZ}{Harmonisch eingeschwungener Zustand}
\newacronym{LL}{LL}{Leerlauf}
\newacronym{KS}{KS}{Kurzschluss}
\newacronym{KSig}{KS}{Kleinsignal}
\newacronym{RZ}{RZ}{Ruhezustand}
\newacronym{AW}{AW}{Anfangswert}
\newacronym{ESB}{ESB}{Ersatzschaltbild}
\newacronym{ESPQ}{ESPQ}{Ersatzspannungsquelle}
\newacronym{ESTQ}{ESTQ}{Ersatzstromquelle}
\newacronym{AWP}{AWP}{Anfangswertproblem}
\newacronym{AP}{AP}{Arbeitspunkt}
\newacronym{AG}{AG}{Arbeitsgerade}
\newacronym{ZRG}{ZRG}{zustandsreduzierende Gleichung}
\newacronym{PWL}{PWL}{piecewise linear}
\newacronym{MNA}{MNA}{Modifizierte Knotenspannungsanalyse}%Modified nodal analysis

\glsaddall
% Skript only
% \newacronym{MDEZ}{MDEZ}{Menge der Darstellungen im eingeschwungenen Zustand}
% \newacronym{MGRF}{MGRF}{Menge der gebrochen raitonalen Funktionen}


\maketitle
\tableofcontents
\newpage


% \chapter{05.11.2010}

\chapter{Frequenzgang/Übertragungsfunktion}
\Index{Frequenzgang}\Index{Übertragungsfunktion}

\begin{skills}
  \item \skillref{stromteiler}{Strom-}/\skillref{spannungsteiler}{Spannungsteiler}
  \item Komplexe Wechselstromrechnung
\end{skills}


\section{\protect\acs{FG}/\protect\acs{UF} ausrechnen:}
\[\cn{H}(j\omega) = \frac{\cn A}{\cn E}\]

\infobox{Faustregeln zum Aufstellen von FG/ÜF:}{
\begin{enumerate}
  \item Ist die Eingangsquelle $e(t)$ eine 
    \begin{tabular}{@{}l@{}}
      \acs{SPQ} \\ \acs{STQ}
    \end{tabular}, nimmt man einen
    \begin{tabular}{@{}l@{}}
      Spg-Teiler \\ Str-Teiler
    \end{tabular}.\\
    Macht man das nicht, muss man eine Gesamtimpedanz/Gesamtadmittanz ausrechen.
  \item Doppelbrüche beseitigen, sodass Zähler und Nenner Polynome in
    $(j\omega)$ (bei \acs{FG}) bzw. $\underline{s}$
    (bei \acs{UF}) sind.
  \item Beim FG:
    $(j\omega)^n$ \underline{\underline{nicht}} ausmultiplizieren !
  \item Parallelschreibweise verwenden!\\
    \[\begin{circuitikz}[scale=0.5,baseline=(current bounding box.center)]
      \draw (0,0)
        to [R=$R$] (0,3)
        -- (2,3)
        to [C=$C$] (2,0)
        -- (0,0);
    \end{circuitikz}
    \underset{\text{KWSR}}{\mathrel{\widehat{=}}}
    R \parallel \frac{1}{j\omega C}
    \]
\end{enumerate}}

\subsection{Stromteiler}\skilllabel{stromteiler}
\Index{Stromteiler}

\begin{minipage}{0.5\textwidth}\centering
\begin{circuitikz}[scale=0.7]
  \draw (0,-0.2)
    to[short,o-] (0,0)
    to[I=$\cn J$] (0,3)
    to[short,-o] (1,3)
    -- (1,2)
    to[R=$\cn Y_1$] (4,2)
    to[short,i_=$\cn{I_1}$] (5,2)
    -- (5,3);
  \draw (1,3)
    -- (1,4)
    to[R=$\cn Y_2$] (4,4)
    -- (5,4)
    -- (5,3)
    to[short,o-o] (6,3);
\end{circuitikz}
\end{minipage}
\begin{minipage}{0.5\textwidth}
\begin{align*}
  \frac{\cn{I_1}}{\cn J}
  &= \colorbox{emphlight}{$\displaystyle\frac{\cn{Y_1}}{\cn{Y_1}+\cn{Y_2}}$}
    \cdot \frac{\cn{Z_1}\cdot\cn{Z_2}}{\cn{Z_1}\cdot\cn{Z_2}}\\
  &= \colorbox{emphlight}{$\displaystyle\frac{\cn{Z_2}}{\cn{Z_1}+\cn{Z_2}}$}
\end{align*}
\end{minipage}
\begin{center}
Nur für 2 parallele $\cn Y$ !
\end{center}

\subsection{Spannungsteiler}\skilllabel{spannungsteiler}
\Index{Spannungsteiler}

\begin{minipage}{0.5\textwidth}\centering
\begin{circuitikz}[scale=0.7]
  \draw (0,-0.2)
    to[short,o-] (0,0)
    to[V=$\cn U$] (0,3)
    to[short] (1,3)
    to[R=$\cn Z_1$] (4,3)
    to[R=$\cn Z_2$] (7,3)
    to[short,-o] (8,3);
\end{circuitikz}
\end{minipage}
\begin{minipage}{0.5\textwidth}
\begin{align*}
  \frac{\cn{U_1}}{\cn U}
  &= \colorbox{emphlight}{$\displaystyle\frac{\cn{Z_1}}{\cn{Z_1}+\cn{Z_2}}$}
\end{align*}
\end{minipage}
\begin{center}
% Nur für 2 serielle $\cn Z$ !
\end{center}

\section{Beispielaufgaben}

\subsection{H07 A1 a)}

\begin{align*}
  \underline{H}(j\omega)
  &=\underbrace{
    \frac{j\omega L_1+R_1}
      {j\omega L_1+R_1+\left(j\omega L_2 + R_2\parallel \frac{1}{j\omega
L_2}\right)}
    }_{\dfrac{\cn{I}_{L_2}}{\cn{J}}}
    \cdot\underbrace{
      \left(-\frac{\frac{1}{j\omega +C_2}}{R_2+\frac{1}{j\omega C_2}}\right)
    }_{\dfrac{\cn{I}_{R_2}}{\cn{I}_{L_2}}}  \\
  &= -\frac{j\omega L_1+R_1}
    {((j\omega)(L_1+L_2)+R_1)\cdot(j\omega R_2C_2+1)+R_2} \\
  &= -\frac{j\omega L_1+R_1}
    {(j\omega)^2R_2C_2(L_1+L_2)+(j\omega)(R_1R_2C_2+L_1+L_2)+R_1+R_2}
\end{align*}


\subsection{H06 A1 a)}

\begin{align*}
  \cn{H}(j\omega)
  &=\underbrace{
    \frac{\cul[green]{R_2\parallel \left(R_3+\frac{1}{j\omega C_2}\right)}}
      {\cul[green]{R_2\parallel \left(R_3+\frac{1}{j\omega C_2}\right)}+\left(R_1\parallel
\frac{1}{j\omega C_1}\right)}
    }_{\dfrac{\cn{U}_{R_2}}{\cn{V}}}
    \cdot\underbrace{
      \frac{\frac{1}{j\omega C_2}}{R_3+\frac{1}{j\omega C_2}}
    }_{\dfrac{\cn{U}_{C_2}}{\cn{U}_{R_2}}}
    \cdot (j\omega C_2) \\
% \left| \cn{I}_{C_2}=j\omega C_2\cn{U}_{C_2} \right. % TODO: einfügen
  &=\frac{R_2\left(R_3+\frac{1}{j\omega C_2}\right)}
    {R_2\cdot\left(R_3+\frac{1}{j\omega C_2}\right)
      +\left(R_2+R_3+\frac{1}{j\omega C_2}\right)
      \cdot R_1\parallel \frac{1}{j\omega C_1}}
    \cdot\frac{j\omega C_2}{j\omega R_3C_2+1} \\
  &=\frac{R_2}
    {R_2\cdot\left(R_3+\frac{1}{j\omega C_2}\right)
      +\left(R_2+R_3+\frac{1}{j\omega C_2}\right)
      \cdot \frac{R_1}{j\omega R_1C_1+1}}
    \cdot\frac{j\omega R_1C_1+1}{j\omega R_1C_1+1}
    \cdot\frac{j\omega C_2}{j\omega C_2}
\end{align*}


\subsection{F05 A1 a)}

\begin{align*}
  \cn{H}(j\omega)
  &=\frac{\cn{I}_{C_2}}{\cn{V}} = j\omega C_2\frac{\cn{U}_{C_2}}{\cn{V}}  \\
  &=j\omega C_2\frac{\cul[green]{\frac{1}{j\omega C_2}\parallel (j\omega L+R)}}
      {\cul[green]{\frac{1}{j\omega C_2}\parallel (j\omega L+R)}+\frac{1}{j\omega C_1}}  \\
  &=\cancel{j\omega C_2}\frac{\frac{1}{\cancel{j\omega C_2}}\cdot(j\omega L+R)}
    {\frac{1}{j\omega C_2}(j\omega L+R)+\frac{1}{j\omega C_1}
      \cdot(j\omega L+R+\frac{1}{j\omega C_2})}
    \cdot\frac{(j\omega)^2 C_1C_2}{(j\omega)^2 C_1C_2} \\
% Nenner wegkürzen!
%   &=\frac{j\omega L+R}
%     {j\omega L+R+\frac{1}{j\omega C_2}((j\omega)^2C_2L+j\omega C_2 R+1)}  \\
  &=\frac{(j\omega)^3C_1C_2L+(j\omega)^2 RC_1C_2}
    {(j\omega)C_1(j\omega L+R)+((j\omega)^2C_2L+(j\omega)RC_2+1)}\\
  &=\frac{(j\omega)^3C_1C_2L+(j\omega)^2 RC_1C_2}
    {(j\omega)^2(C_1+C_2)L+(j\omega)R(C_1+C_2)+1}
\end{align*}


\subsection{H05 A1 a)}

\begin{align*}
  \cn{H}(j\omega)
  &=\frac{\cn{I}_L}{\cn{J}}
    =\frac{R_1}
    {R_1+\left(j\omega L+R_2\parallel \frac{1}{j\omega C}\right)}
    \cdot\frac{(j\omega R_2C+1)}
      {(j\omega R_2 C+1)} \\
  &=\frac{R_1(j\omega R_2C+1)}
    {(R_1+j\omega L)(j\omega R_2C+1)+R_2} \\
  &=\frac{(j\omega)\cdot R_1R_2C+R_1}
    {(j\omega)^2 R_2CL+(j\omega)\cdot(R_1R_2C+L)+R_1+R_2}
\end{align*}

Allgemeiner Hinweis: \colorbox{emphlight}{$\displaystyle
  R\parallel \frac{1}{j\omega C} = \frac{R}{j\omega RC+1}$}


\subsection{F06 A1 a)}

\begin{align*}
  \cn{H}(j\omega)
  &=-j\omega L_2
    \frac{R_1+j\omega L_1}
      {R_1+j\omega L_1+R_2+j\omega L_2\parallel R_3}
    \cdot\frac{R_3}
      {j\omega L_2+R_3} \\
  &=-\frac{(j\omega)^2R_3L_2L_1+(j\omega)R_1R_3L_2}
    {(j\omega L_1+R_1+R_2)(j\omega L_2+R_3)+j\omega R_3L_2} \\
  &=-\frac{(j\omega)^2R_3L_1L_2+(j\omega)R_1R_3L_2}
    {(j\omega)^2L_1L_2+(j\omega)(L_1R_3+(R_1+R_2+R_3)L_2)+(R_1+R_2)R_3}
\end{align*}

% TODO: Unterstreichungen?
 % FG/ÜF
% \chapter{12.11.2010}

\chapter{Asymptotische Stabilität}

\infobox*{\raggedright%
  Ein Netzwerkmodell ist asymptotisch stabil, wenn
  alle natürlichen Frequenzen des Netzwerkmodells einen
  \emph{negativen Realteil} haben.\\
  \raggedleft\small($\to$ Satz 5.2)}
Die nat. Freq. $\cn A_i$ bestimmen, ob, und wenn ja,
wie schnell das \acs{NWM} einschwingen \emph{kann}.

% \begin{itemize}
%     $\to$ Bsp.: Die Lösung des \acs{NWM}s liefert $e^{\cn{A}_i\cdot t}$,
%     $\cn{A}_i$: natürliche Frequenz, $\cn{A}_i = \alpha + j\beta$
%     \begin{description}
%       \item[NR:] $e^{\cn{A}_i t} = e^{\alpha t}\cdot e^{j\beta t}$;
%         $e^{\alpha t}$ klingt ab, wenn $\alpha<0$
%     \end{description}
% \end{itemize}


\section{Wie beweist man asymptotische Stabilität?}% margin: siehe FS

Zuerst müssen \emph{alle} ($n$) natürlichen Frequenzen gefunden werden (Punkt 1-3).
Dann wird bewiesen, dass sie einen negativen Realteil haben (Hurwitz-Kriterium)
(Punkt 4).



\begin{enumerate}
  \item $n_A$ und $n_R$ bestimmen!\hfill
  \begin{description}
    \item[$n_A:$] Anzahl der differenzierbaren Variablen,
      also Variablen, die abgeleitet werden.
      \begin{description}
        \item[Bsp.:]\hfill\\
          \begin{minipage}{0.5\textwidth}
            \begin{circuitikz}[scale=0.7,baseline=(current bounding box.center)]
              \ctikzset{voltage/distance from node=.3}
              \draw (0,0)
                to[C=$C$,i=$i_C(t)$,v>=$u_C(t)$] (0,-3);
            \end{circuitikz}
            $i_c(t) = C\cdot \frac{\partial}{\partial t}\colorbox{emphlight}{$u_C(t)$}$
          \end{minipage}
          \begin{minipage}{0.5\textwidth}
            \begin{circuitikz}[scale=0.7,baseline=(current bounding box.center)]
              \ctikzset{voltage/distance from node=.4}
              \draw (0,0)
                to[L=$L$,i=$i_L(t)$,v>=$u_L(t)$] (0,-3);
            \end{circuitikz}
            $u_L(t) = L\cdot \frac{\partial}{\partial t}\colorbox{emphlight}{$i_L(t)$}$
          \end{minipage}
          \par\vspace*{5mm}
          \begin{minipage}{0.5\textwidth}
            ~\,%
            \begin{circuitikz}[scale=0.7,baseline=(current bounding box.center)]
              \ctikzset{voltage/distance from node=.4}
              \draw (0,0)
                to[R=$R$,v>=$u_R(t)$,i={\,}] (0,-3);
              \draw (2,0)
                to[I] (2,-3);
            \end{circuitikz}
            $\alpha \frac{\partial}{\partial t} \colorbox{emphlight}{$u_R(t)$}$
          \end{minipage}
          \begin{minipage}{0.5\textwidth}
            \quad%
            \begin{circuitikz}[scale=0.7,baseline=(current bounding box.center)]
              \ctikzset{voltage/distance from node=.4}
              \draw (0,0)
                to[L=$L$,i>=$i_R(t)$] (0,-3);
              \draw (2,0)
                to[V] (2,-3);
            \end{circuitikz}
            $\beta\frac{\partial}{\partial t} \colorbox{emphlight}{$i_L(t)$}$
          \end{minipage}
      \end{description}
%       \begin{itemize}
%         \item Erst $u_C(t)$, $i_L(t)$ zählen
%         \item Dann die durch Ableitung gesteuerten Quellen.\\
%           Diese liefern nur
%           dann eine differenzierbare Variable, falls sie noch nicht gezählt wurden.\\
%           $\to$ \emph{Keine differenzierbare Variable wird "`doppelt"' gezählt}
%       \end{itemize}

    \item[$n_R:$] Anzahl der zustandsreduzierenden Gleichungen (ZRG)
      \begin{description}
        \item[Allgemein:]
          \[\sum_{i=1}^n \alpha_i\cdot x_i(t)
            + \sum_{i=1}^{n_Q} \beta_i\cdot e_i^{(k)}(t) = 0\]
          \begin{tabular}{r@{\,:~}l}
            $x_i(t)$      &differenzierbare Variable  \\
            $e_i^{(k)}$   &\emph{ungesteuerte} Quellen + deren Ableitungen \\
            $\alpha_i, \beta_i$ &Konstanten, z.B. $RCL$, etc.
          \end{tabular}
        \item[Spezialfälle:]\hfill
          \begin{itemize}
            \item MGL \emph{nur} aus $C$s und/oder festen SPQ
            \item SGL \emph{nur} aus $L$s und/oder festen STQ
          \end{itemize}
          \leftpointright~Gibt es \emph{keine} gesteuerten Quellen im NWM,
          kann es nur solche ZRG geben.
      \end{description}
      Wichtig: Für die Anzahl $n_{R_g}$ der \emph{gefundenen} ZRG gilt:
      \[n_R \geq n_{R_g}\qquad !\]
    \item[$n:$]\hfill
      \begin{itemize}
        \item Ordnung des Netzwerkmodells, $n = n_A-n_R$
        \item Anzahl an nat. Freq. des NWMs
          (die man auf neg. Realteil testen soll)
        \item Ordnung des DGL-Systems, also ZRM (Zustandsraummodell)
      \end{itemize}
    \item[$\Rightarrow$] \colorbox{emphlight}{$n = n_A-n_R$}
  \end{description}

  \item ÜF/FG aufstellen:
    \[\cn{H}(\cn{s}) = \frac{\cn{P}(\cn{s})}{\cn{Q}(\cn{s})}\ ;\quad
      \cn{P}(\cn{s}), \cn{Q}(\cn{s})\text{ Polynome}
    \]
    Sind $\cn{P}(\cn{s})$, $\cn{Q}(\cn{s})$ teilerfremd?\\
    \begin{tabular}{@{$\rightarrow$\,}r@{\,:~\,}l@{}}
      Ja    & weiter, alle Nullstellen von $\cn{Q}(\cn{s})$ sind
        natürliche Frequenzen\\
      Nein  & kürzen
    \end{tabular}
    \begin{description}
      \item[Anmerkung:] Hat man mehrere ÜF, müssen auch die $\cn{Q}_i(\cn{s})$
        zueinander teilerfremd sein
    \end{description}

  \item Es gilt $n_A-n_R = n \geq \grad(\cn{Q}(\cn{s}))$.
    Gilt auch $n_A-n_R=\grad(\cn{Q}(\cn{s}))$?\\
    \begin{tabular}{@{$\rightarrow$\,}r@{\,:~\,}l@{}}
      Ja    & weiter  \\
      Nein  & Neue ÜF aufstellen, die (hoffentlich)
        alle fehlenden nat. Freq. (zu 2.) liefert
    \end{tabular}

  \item Haben alle Nullstellen von $\cn{Q}(\cn{s})$ (bzw. $\cn{Q_i}(\cn{s})$)
    einen negativen Realteil?\\
    \begin{tabular}{@{$\rightarrow$\,}r@{\,:~\,}l@{}}
      Ja    & Das NWM ist asymptotisch stabil  \\
      Nein  & Das NWM ist instabil
    \end{tabular}
    \begin{description}
      \item[Anmerkung:]
        $\cn{Q}(\cn{s})$ heißt \emph{Hurwitzpolynom}, falls alle Nullstellen
        einen negativen Realteil haben.
        \Index{Hurwitzpolynom}
        
        Für $\grad(\cn{Q}(\cn s)) \leq 2$ gilt:
        \begin{empheq}[box=\colorbox{emphlight}]{align*}
          \text{Alle Koeffizienten sind }\begin{matrix}\text{positiv}\\[-0.5ex]\text{negativ}\end{matrix}
            &\Leftrightarrow \cn{Q}(\cn{s})\text{ ist ein Hurwitzpolynom }\\
            &\Leftrightarrow\text{Alle Nullstellen haben negativen Realteil}
          \end{empheq}
    \end{description}

\end{enumerate}

% TODO:
% \subsection{Teilerfremdheit}
% Einfacher ist es, Zähler-Nullstellen in den Nenner einzusetzen.
% Beim mehreren Q(s) testen, ob Q(s)s untereinander teilerfremd.


% \begin{description}
%   \item[Anmerkung:] Bei mehreren \acs{ZRG} muss man prüfen,
%     ob sie \emph{linear unabhängig} sind\\
%     $\to$ ob eine davon durch die anderen darstellbar ist
% \end{description}

\clearpage
\section{Beispiel: F08 A1}

\begin{figure}[h]
\begin{center}
\begin{circuitikz}[scale=1.2]\draw
	(0,0) -- (6,0)
	to [C,l_=$C_2$,v^>=$u_{C_2}(t)$] (6,-4)
	-- (4,-4)
	to [L=$L$, *-] (4,-2)
	to [R,l_=$R_2$,v^>=$u_{R_2}(t)$, -*] (4, 0)
	(4,-4)
	-- (2,-4)
	to[R=$R_1$,*-] (2,-2)
	-- (0,-2)
	to [C=$C_1$,v=$u_{C_1}(t)$,*-] (0,-4)
	-- (2,-4)
	(0,-2)
	to [V<=$v(t)$] (0,0);
\draw (2,-4) to [very thick] (4,-4);
\end{circuitikz}
\end{center}
\caption*{$R_i = R$, $C_i = C$}
\end{figure}

\paragraph{Aufgabe}
\begin{enumerate}[label=\alph{*})]
	\item $\cn{H}(j\omega) = \frac{\cn{P}(j\omega)}{\cn{Q}(j\omega)} =
\frac{\cn{U}_{R_2}}{\cn{V}}$
	\item Ist das \acs{NWM} asymptotisch stabil?
\end{enumerate}

% \infobox{Beispiel für zustandsreduzierende Gleichungen}{%
% 
% \begin{compactitem}
% 	\item Maschen aus $C$ und/oder festen SPQ
% 	\item Schnittmengen aus L und/oder festen STQ
% \end{compactitem}
% 
% \paragraph{Allgemein:}
% 
% $x_1(t), \hdots, x_n(t)$ sind diffbare Variablen,
% $e_1(t), \hdots, e_{n_Q}(t)$ sind feste Quellen
% \[
% \Rightarrow\quad \sum_{i=1}^n \alpha_i \cdot x_i(t) + \sum_{i=1}^{n_Q}\sum_{k=0}^{r_i} e_i^{(k)}(t) \cdot \beta_{i,k} = 0
% \]
% ist eine allgemeine zustandsreduzierende Gleichung.
% }

\paragraph{Lösung}
\begin{enumerate}[label=\arabic{*})]

  \item $n_A$, $n_R$ aufstellen:\par
  $n_A\colon n_A=3$, denn $u_{C_1}(t), u_{C_2}(t), i_L(t)$
    sind diffbare Variable!\\
  $n_R\colon n_R\geq 1$, wegen Masche aus $C_1, C_2, v(t)$% TODO: n_R\geq 1?

  \acs{MGL}: $u_{C_1}(t) + u_{C_2}(t) - v(t) = 0$
  \begin{description}
    \item[Beispiel:] AW von $u_{C_1}(t)$ für $t = t_0$ vorgegeben\\
      $\Rightarrow u_{C_2}(t_0) = v(t_0) - u_{C_1}(t_0)$
      ist \emph{nicht} frei vorgebbar! (1 Zustand fällt weg!)
  \end{description}
  $\Rightarrow n = n_A- n_R \leq 2$

  \item $\cn H(\cn s)$ aufstellen. Zeigen, dass $\cn{P}(\cn s)$, $\cn{Q}(\cn s)$
    teilerfremd sind:
    %
    \[\text{Aus a) berechnet:}\quad
      \cn H(\cn s) = \frac{\cn sCR^2+R}{\cn s^2 2RCL+\cn s\left(2R^2C+L\right)+2R}\]
    Zählerpolynom zu Null setzen und Ergebnis in Nennerpolynom einsetzen:
    \begin{align*}
      \cn{P}(\cn s) &\stackrel{!}{=} 0 \Rightarrow \cn s_0 = -\frac{1}{RC}\\
      \quad\cn{Q}\left(-\frac{1}{RC}\right)% TODO: \cn s -> \cn s !?
      &= \left(-\frac{1}{RC}\right)^2\cdot 2RCL +
        \left(-\frac{1}{RC}\right)\cdot (2R^2C+L) + 2R\\
      &= \frac{2L}{RC} - \frac{L}{RC} - 2R+2R
        = \frac{L}{RC} \underset{R,C,L > 0}{>} 0
    \end{align*}
    $\Rightarrow$ $\cn{P}(\cn s)$, $\cn{Q}(\cn s)$ sind teilerfremd,
    $\cn{Q}(\cn s)$ liefert 2 natürliche Frequenzen!


  \item Gilt $n=\grad(\cn Q(\cn s))?$\par
    $\Rightarrow 2\geq n_A - n_R = n \geq \grad\left(\cn{Q}(\cn s)\right) = 2$

    $\Rightarrow n = \grad\left(\cn{Q}(\cn s)\right) = 2, \cn{Q}(\cn s)$
    liefert \emph{alle} natürlichen Frequenzen des NWM!

  \item Haben alle NS von $\cn Q(\cn s)$ einen negativen Realteil?\par
    $\cn{Q}(\cn s)$ ist ein Hurwitzpolynom 2. Grades,
    daher haben alle Nullstellen einen echt negativen Realteil,
\end{enumerate}
$\rightarrow$ Das NWM ist \textbf{asymptotisch stabil}. \qed
 % asympt. Stabilität
% \chapter{19.11.2010}

\chapter{Anfangswerte}

\emph{Gibt es bei $t=t_s$ inkonsistente AW?}

$\to$ Das bedeutet, man kann AW vorgeben, die beim Schalten \emph{NICHT}
stetig übernommen werden.
Das bedeutet aber nur, dass es mind. einen AW-Vektor gibt, der beim Schalten
nicht stetig übernommen wird.

\paragraph{Beispiel:}\hfill\\

\begin{minipage}{0.5\textwidth}
\begin{circuitikz}
  \draw (1,0)
    to[short,o-] (0,0)
    -- (0,-1)
    to[V=$V_0$] (0,-4)
    -- (4,-4)
    to[C=$C$, v=$u_{C2}(t)$] (4,-2)
    to[C=$C$, v=$u_{C1}(t)$] (4,0)
    to[short,-o] (2,0);
  \draw (1,0) -- +(30:1);
\end{circuitikz}
\end{minipage}
\begin{minipage}{0.5\textwidth}
  \begin{itemize}
    \item Bei $u_{C_1}(t_s^-) = u_{C_2}(t_s^-) = 2V$, $V_0=5V$ ist die MGL
      für $t>t_s$ nicht erfüllt.\\
      $\Rightarrow$ $\underbrace{u_{C_1}(t_s^-)}_{\text{\parbox{1.75cm}{direkt \emph{vor} dem Schalten}}}
      \neq  \underbrace{u_{C_1}(t_s^+)}_{\text{\parbox{1.75cm}{direkt \emph{nach} dem Schalten}}}$
    \item Bei $u_{C_1}(t_s^-) = u_{C_2}(t_s^-) = 2,5V$, $V_0=5V$ ist die MGL
      für $t>t_s$ erfüllt.\\
      $\Rightarrow$ $u_{C_1}(t_s^-) = u_{C_1}(t_s^+) = 2,5V$\\
      Dies gilt aber eben \emph{NICHT} für \emph{alle} AW!
  \end{itemize}
\end{minipage}\medskip



\begin{description}
  \item[Vorgehensweise:]\hfill
    \begin{itemize}
      \item $n_A$, $n_R$ bestimmen für $t>t_s$
      \item zu beweisen: $n_R=0$ für $t>t_s$
        $\Rightarrow$ es gibt keine inkonsistenten AW bei $t=t_s$!\\
        ($n_R>0$: es gibt inkonsistente AW!)
        
      \begin{description}
        \item[Wiederholung]
          Gibt es keine gesteuerten Quellen im NWM, kann es nur 2 Typen von 
          zustandsreduzierenden Gleichungen geben:
          \begin{enumerate}
            \item MGL nur aus $C_s$ und/oder festen SPQ
            \item SGL nur aus $L_s$ und/oder festen STQ
          \end{enumerate}
      \end{description}
    \end{itemize}
\end{description}


\paragraph{Weiteres Beispiel}

[BEISPIEL]% TODO

\paragraph{Bsp zur diffbaren Variable}

[BEISPIEL]% TODO

\paragraph{Bei gesteuerten Quellen (wieder für $t>t_s$)}

\begin{itemize}
  \item $n_A$, $n_R$ bestimmen, wenn $n_R>0$: fertig!
  \item $\cn{H}(j\omega)$,
    $\cn{H}(\cn{s}) = \frac{\cn{P}(\cn{s})}{\cn{Q}(\cn{s})}$ aufstellen.\medskip
    
    $\cn{P}(\cn{s})$, $\cn{Q}(\cn{s})$ teilerfremd? Wenn nein: Polydiv!
  \item $n_A = \grad{(\cn{Q}(\cn{s}))}$? Wenn ja: fertig, da aus
    $n_A  = \grad{(\cn{Q}(\cn{s}))}$ folgt: $n=n_A$, $n_R=0$\\
    $\Rightarrow$ keine zustandsreduzierende Gleichung für $t>t_s$\\
    $\Rightarrow$ keine inkonsistenten AW bei $t=t_s$!
\end{itemize}


\klausurhinweis{
  Wenn sowohl Beweis für asymptotische Stabilität als auch
  für inkonsistente Anfangswerte verlangt wird:
  \begin{enumerate}
    \item Stabilität beweisen
    \item Mit den Ergebnissen argumentieren, ob es inkonsistente AW gibt oder nicht!
  \end{enumerate}
}

\section{Vorgehensweise Anfangswerte}

\begin{enumerate}
  \item Sind alle AW $x(t_s^+)$ gegeben
    (z.\,B. $u_C(t)$, $i_L(t)$, alle differenzierbare Variable
    für die ein AW vorgegeben werden kann)
  \item Sind alle AW für $t=t_s^-$ vorgegeben?
  \item Prüfe: ist das NWM für $t=t_s^-$ eingeschwungen, d.\,h.
    \begin{enumerate}
      \item beweise: Asymptotische Stabilität!
      \item Wartebedingung
    \end{enumerate}
  \item Das NWM ist eingeschwungen:
    \begin{itemize}
      \item Harmonisch eingeschwungener Zustand (HEZ)\\
        $\Rightarrow$ KWSR
      \item DC-Eingeschwungener Zustand (DC-EZ)\\
        $\Rightarrow$ DC-ESB
      \item Allgemein EZ\\
        $\Rightarrow$ Faltung\\
        Bsp.: Rechkeck-, Dreieck-Impuls,
    \end{itemize}
\end{enumerate}



 % Anfangswerte
% \chapter{26.11.2010}

\chapter{Harmonisch Eingeschwungener Zustand (HEZ)}

\begin{skills}
  \item Rechnen mit komplexen Zahlen
  \item Komplexe Wechselstromrechnung
  \item Additionstheoreme
\end{skills}


\paragraph{Voraussetzungen}, damit sich ein \acs{NWM} im \acs{HEZ} befindet:

\klausurhinweis{
  In der Klausur \emph{muss} bewiesen werden, dass sie erfüllt sind!
  Sonst gibt es massiv Punktabzug!
}

\begin{itemize}
  \item asymptotische Stabilität, d.\,h. alle natürlichen Frequenzen $\cn{A}_i$
    haben $\Re{\cn{A}_i} < 0$
  \item Das \acs{NWM} muss sich im eingeschwungenen Zustand befinden, \mbox{d.\,h.}\xspace
    $t-t_s$, $t-t_n >> \underset{1\leq i\leq n}{\operatorname{max}}
    \left(-\frac{1}{\Re{\cn{A}_i}}\right)$, die $-\frac{1}{\Re{\cn{A}_i}}$
    sind die Zeitkonstanten des \acs{NWM}s
  \item Für alle festen Quellen $e_k(t)$ gilt:
  
    $e_k(t)$ ist rein harmonisch für $t>t_k$, $k=1,\ldots,n_Q$, dabei ist
    $t_m = \underset{1\leq i\leq n_Q}{\operatorname{max}}(t_k)$,
    d.h. $t_m$ ist der Zeitpunkt, ab dem alle Quellen rein harmonisch sind!
\end{itemize}

% TODO: Zeichnung

\paragraph{Folgerungen:}
\begin{itemize}
  \item Alle Anfangswerte sind abgeklungen ($\rightarrow$EZ), $a_{zi}(t)=0$
  
  \item Alle Zweiggrößen sind rein harmonische Funktionen:\\
    $a(t)=\Re{\cn{A}\cdot e^{j\omega t}}
    \underset{\cn{A}=A\cdot e^{j\varphi_A}}{=}
    A\cos(\omega t+\varphi_A)$
    
    $\Rightarrow$ Das NWM lässt sich vollständig durch \acs{KWSR} beschreiben!
    
    \paragraph{Beispiel}
    
    \begin{align*}
      u_R(t) &= R\cdot i_R(t) 
      &&\underset{
        \begin{aligned}
          i_R(t)
            &= I_R\cdot\cos(\omega t+\varphi)  \\
            &= \Re{I_R\cdot e^{j\varphi}\cdot e^{j\omega t}}  \\
            &= \Re{\cn{I}_R\cdot e^{j\omega t}}
        \end{aligned}
      }{\xrightarrow{\hspace{1cm}}}
      &&\begin{aligned}
        u_R(t)
          &= \Re{R\cdot \cn{I}_R\cdot e^{j\omega t}} \\
          &= \Re{\cn{U}_R\cdot e^{j\omega t}}
      \end{aligned} \\
      %
      u_L(t) &= L\cdot \frac{\partial}{\partial t}i_L(t) 
      &&\underset{
        \begin{aligned}
          i_L(t) = \Re{\cn{I}_L\cdot e^{j\omega t}}
        \end{aligned}}
        {\xrightarrow{\hspace{1cm}}}
      &&\begin{aligned}
          u_L(t)
            &= \Re{L\cdot \frac{\partial}{\partial t} \cn{I}_L
              \cdot e^{j\omega t}}  \\
            &= \Re{j\omega L\cdot\cn{I}_L\cdot e^{j\omega t}} \\
            &= \Re{\cn{U}_L\cdot e^{j\omega t}}
        \end{aligned} \\
      %
      i_c(t) &= C\cdot \frac{\partial}{\partial t} u_C(t)
      &&\underset{
        \begin{aligned}
          u_C(t) = \Re{\cn{U}_C\cdot e^{j\omega t}}
        \end{aligned}}
        {\xrightarrow{\hspace{1cm}}}
      &&\begin{aligned}
          i_c(t)
            &= \Re{C\cdot \frac{\partial}{\partial t}\cn{U}_C
              \cdot e^{j\omega t}}  \\
            &= \Re{j\omega C\cdot \cn{U}_C\cdot e^{j\omega t}}
        \end{aligned}
    \end{align*}

    Das Einzige, was sich bei den \acs{ZGL} ändert, sind die Zeiger:
    \[\left.
      \begin{aligned}
        \cn{U}_R &= R\cdot \cn{I}_R  \\
        \cn{U}_L &= j\omega L\cdot \cn{I}_L  \\
        \cn{I}_C &= j\omega C\cdot \cn{U}_c
      \end{aligned}
      \right\}\text{ZGL der komplexen Wechselstromrechnung}\]
    (Für MGL/SGL geht es genauso!)
\end{itemize}

\clearpage
\section{Vorgehensweise HEZ}

\begin{enumerate}
%   \item Stelle $e(t)$ bzw. $e_\infty(t)$ auf in der Form:
%     \[e(t)=\sum_{k=1}^n E_k \cos(\omega_k t+\varphi_k)\qquad
%     \text{//$n$ ist hier \emph{nicht} die Ordnung des NWM!}\]
  \item Stelle $e(t)$ bzw. $e_\infty(t)$ auf in der Form:
    \[\colorbox{emphlight}{$\displaystyle e(t)=\sum_{k=1}^n E_k \cdot\cos(\omega_k t+\varphi_k)$}\]
    Dazu muss $e(t)$ ggf. mit Hilfe von Additionstheoremen umgeformt werden.
    Alle $E_k$, $\omega_k$, $\varphi_k$ sind zu definieren!\\
    Hilfreich: $V=V\cdot \cos(0\cdot t+0)$, falls $v(t) = V$

  \item Beweise, dass die Vorraussetzungen für HEZ erfüllt sind
    (Antwortsatz: Formelsammlung)

  \item Stelle $\cn H(j\omega)$ in Polarkoordinaten dar
    ($\cn H(j\omega)=\frac{\cn P(j\omega)}{\cn Q(j\omega)}$ mit $\cn P(j\omega)$,
    $\cn Q(j\omega)$ Polynome):
    \begin{align*}
      \cn{H}(j\omega)&\phantom{:}=:K(\omega)
        \cdot e^{j(\varphi_P(\omega)-\varphi_Q(\omega))}\\
      K(\omega) &:= |\cn{H}(j\omega)|
        = \sqrt{\frac{\Im{\cn{P}(j\omega)}^2 + \Re{\cn{P}(j\omega)}^2}
        {\Im{\cn{Q}(j\omega)}^2 + \Re{\cn{Q}(j\omega)}^2}}
        \qquad \text{// nicht vereinfachen!}\\
      \varphi_{P}(\omega)
        &:= \arctan\left({\frac{\Im{\cn P(j\omega)}}{\Re{\cn P(j\omega)}}}\right) +
        \begin{cases}
          0   &\text{falls} \Re{\cn P(j\omega)} > 0 \\
          \pi &\text{falls} \Re{\cn P(j\omega)} < 0
        \end{cases}\\
        % TODO
        % Muss jür \emph{jedes} $\omega_k$ einzeln geprüft werden, sodass
        % danach üfr jedes $\omega_k$ klar ist, zu welchem Fall es gehört
      \varphi_Q(\omega) &\phantom{:=} \text{analog zu } \varphi_P(\omega)
    \end{align*}
    Damit gilt (für \emph{eine} Quelle mit \emph{einer} Frequenz):
    \[\cn{A}=\cn{H}(j\omega)\cdot \cn{E}
      = E\cdot K(\omega)
        \cdot e^{j(\varphi_E+\varphi_P(\omega)-\varphi_Q(\omega))}\]
    Transformation in den Zeitbereich:
    \[\Rightarrow a(t) = \Re{\cn{A}\cdot e^{j\omega t}}
      = E \cdot K(\omega)
        \cdot \cos(\omega t+\varphi_E+\varphi_P(\omega)-\varphi_Q(\omega))\]


  \item Berechne $a(t)$ (näherungsweise) per Superposition im HEZ:
    \[a(t) \approx a_{\text{hez}}
      = \sum_{k=1}^{n}
        E_k\cdot K(\omega_k)\cdot\cos(
          \omega_k t+\varphi_{E,k}+\varphi_P(\omega_k)-\varphi_Q(\omega_k))\]
    $\varphi_P(\omega_k)$ und $\varphi_Q(\omega_k)$ müssen für jedes $k$
    bestimmt werden!
\end{enumerate}


%     Zuerst die Voraussetzungen beweisen, dann:
%     \[\cn{H}(j\omega) = \frac{\cn{A}}{\cn{E}}
%       = \frac{\cn{P}(j\omega)}{\cn{Q}(j\omega)}
%       \text{ in Polarkoordinaten darstellen: }
%       \cn{H}(j\omega):=K(\omega)
%         \cdot e^{j(\varphi_P(\omega)-\varphi_Q(\omega))} % TODO: check!
%       \]
%     D.h. Betrag ($K(\omega)$) und Argument ($\varphi_{P,Q}$) definieren:
%     \[\]
%     \[\varphi_{P,Q}(\omega)
%       := \arctan\left({\frac{\Im{}}{\Re{}}}\right) +
%         \begin{cases}
%           0   &\text{falls} \Re{} > 0 \\
%           \pi &\text{falls} \Re{} < 0
%         \end{cases}
%       \qquad \text{definieren!}
%     \]


\section{Komplexe Zahlen}

\begin{align*}
  \cn z = a+jb = r\cdot e^{j\varphi} = r\cdot (\cos\varphi + j\sin\varphi)
\end{align*}


\subsection{Betrag einer komplexen Zahl}\skilllabel{komplexbetrag}

\[|\cn z| = \sqrt{\Im{\cn z}^2+\Re{\cn z}^2}\]

\subsection{Winkel einer komplexen Zahl}\skilllabel{komplexwinkel}\hfill

\begin{minipage}{0.5\textwidth}
\begin{tikzpicture}[scale=0.8]
  \draw[->,>=latex] (-3,0) -- (3,0) node [below right] {$\operatorname{Re}$};
  \draw[->,>=latex] (0,-3) -- (0,3) node [left] {$\operatorname{Im}$};
  
  \coordinate[] (P) at (2,2) {};
  \coordinate[] (Ps) at (-2,-2) {};
  \draw[fill] (P) circle (0.05);
  \draw[fill] (Ps) circle (0.05);
  \draw (Ps) node[left] {$-\cn{z}$}
    -- (P) node [right] {$\cn{z} = x+jy\text{ mit }x,y>0$};
  \draw[blue, thick] (0,0) -| (P);
  \draw (1,0) arc (0:45:1) node [below=0.2] {$\varphi$};
\end{tikzpicture}
\end{minipage}
\hfill
\begin{minipage}{0.4\textwidth}
\[\color{blue}\left(\text{Hauptwert des $\arctan$: }
  \real\to\left(-\frac{\pi}{2};\frac{\pi}{2}\right)
  \right)\]
\end{minipage}

Für den Winkel gilt:
\[\angle \cn{z} = \arctan\left(\frac{y}{x}\right)
  \in \left(-\frac{\pi}{2};\frac{\pi}{2}\right) \]
für $-\cn{z}$ gilt: $\angle-\cn{z} = \angle z \pm\pi$,
gleichzeitig liefert der Hauptwert des $\arctan()$:
\[\arctan{\left(\frac{-y}{-x}\right) = \arctan\left(\frac{y}{x}\right)}
  =\angle\cn{z}\]
Der Hauptwert liefert Werte zischen $-\frac{\pi}{2}$ und $\frac{\pi}{2}$
zurück, er berechnet den Winkel von $\cn{z}$ korrekt, solange
\[\angle\cn{z} \in \left(-\frac{\pi}{2};\frac{\pi}{2}\right)
  \Leftrightarrow \cn{z} \text{ liegt in der rechten \acs{HE} }
  \Leftrightarrow \Re{\cn z} > 0\]
Andernfalls  muss noch die Korrektur "`$+\pi$"' dazuaddiert werden:
\[\colorbox{emphlight}{$\angle\cn{z} = \arctan\left(\frac{y}{x}\right)
  + \begin{cases}
      0 & \text{falls } \Re{\cn{z}}=x > 0 \text{ \quad// rechte \acs{HE}} \\
      \pi & \text{falls } \Re{\cn{z}}=x < 0 \text{ \quad// linke \acs{HE}}
    \end{cases}$}
\]



% \section{Erweiterung der Rechentechniken auf mehrere Quellen/eine Quelle
% mit mehreren Frequenzen}
% 
% \begin{itemize}
%   \item Forme $e(t)$ mit Hilfe von Additionstheoremen solange um, dass $e(t)$
%     folgende Form hat:
%     \[e(t) = \sum_{k=1}^{n} E_k\cdot \cos(\omega_k\cdot t+\varphi_k)
%       \qquad\begin{aligned}
%         &\text{// Hilfreich: } V=V\cdot \cos(0\cdot t+0)
%         \text{, falls } v(t) = V \\
%         &\text{// Alle } E_k, \omega_k, \varphi_k \text{ definieren!}
%       \end{aligned}\]
%   \item Per Superposition gilt:
%     \begin{align*} 
%       a_{hez}(t) &\;\;\;= \Re{
%         \sum_{k=1}^{n} \underbrace{\underbrace{
%           E_k\cdot e^{j\varphi_k}}_{\cn{E}_k}
%         \cdot \cn{H}(j\omega_k)}_{\cn{A}_k=\cn{H}(j\omega)\cdot\cn{E}_k}
%         \cdot e^{j\omega_k t}} \\
%       &\underset{\begin{matrix}\text{Polar-}\\\text{darst.}\end{matrix}}{=}\Re{
%         \sum_{k=1}^{n} E_k \cdot K(\omega_k)
%           \cdot e^{j(\omega_k\cdot t +\varphi_k
%             + \varphi_P(\omega_k) - \varphi_Q(\omega_k))}}  \\
%       &\;\;\;= \sum_{k=1}^{n}
%         E_k\cdot K(\omega_k)\cdot\cos(
%           \omega_k\cdot t+\varphi_k+\varphi_P(\omega_k)-\varphi_Q(\omega_k))
%     \end{align*}
%   \item \color{blue} Berechne für $e_k(t)$ den Anteil $a_{hez,k}(t)$ wie oben
%     für alle $k=1,\ldots,n_Q$.
%     
%     Dann gilt wieder per Superposition:
%     \[a_{hez}(t) = \sum_{k=1}^{n_Q} a_{hez,k}(t)\]
% \end{itemize}



% \chapter{03.12.2010}


% TODO: ESB für t > t_s
\section{Beispielaufgabe F05 A1 a),b)}

\begin{enumerate}[label= \bfseries\alph*)]
  \item 
    \begin{align*}
      \cn{H} &\;\;= \frac{\cn{I}_{C_2}}{\cn V}
        = j\omega C_2\cdot\frac{\cn{U}_{C_2}}{\cn V}  \\
        &\underset{\begin{matrix}
            \text{\footnotesize Spg.-}\\
            \text{\footnotesize Teiler}
          \end{matrix}}{=}
          j\omega C_2\cdot\frac{\frac{1}{j\omega C_2}\parallel(j\omega L+R)}
          {\frac{1}{j\omega C_2}\parallel(j\omega L+R)+\frac{1}{j\omega C_1}}\\
        &\;\;= \cancel{j\omega C_2}
          \cdot\frac{\cancel{\frac{1}{j\omega C_2}}\cdot (j\omega L+R)}
            {\frac{1}{j\omega C_2} \cdot (j\omega L+R) + \frac{1}{j\omega C_1}
              \cdot\left(j\omega L+R+\frac{1}{j\omega C_2}\right)}
          \cdot\frac{(j\omega)^2C_1C_2}
            {(j\omega)^2C_1C_2} \\
        &\;\;=\frac{(j\omega)^3C_1C_2L+(j\omega)^2C_1C_2R}
          {(j\omega)^2(C_1+C_2)L+(j\omega)\cdot R(C_1+C_2)+1}
    \end{align*}

  \item
    \begin{itemize}
      \item Stelle $e(t) = v(t)$ als Summe von $\cos$-Fkt. dar:
        \[v(t) = \underbrace{A\cdot \sin(\omega_0 t+\varphi_1)}_{
            \begin{aligned}
              E_1       &:= A \\
              \omega_1  &:= \omega_0  \\
              \Phi_1    &:= \varphi_1-\frac\pi2
            \end{aligned}}
          +\underbrace{B\cdot \cos(3\omega_0 t)}_{
            \begin{aligned}
              E_2       &:= B \\
              \omega_2  &:= 3\omega_0 \\
              \Phi_2    &:= 0
            \end{aligned}}
          =\sum_{k=1}^{2} E_k\cdot \cos(\omega_k t+\Phi_k)\]
      \item Voraussetzung für den HEZ erfüllt?
      
        \begin{itemize}[label=$\rightarrow$]
          \item Asympt. Stabilität!
          \item $t-t_s$, $t-t_m >> \underset{i}{\operatorname{max}}
            \left\{-\frac{1}{\Re{\cn{A}_i}}\right\}$
          \item Alle $e(t)$ für $t>t_m$ sind harmonisch!
        \end{itemize}
        
        \underline{$\Rightarrow$ Erfüllt.}
        
        Das als asympt. stabil angenommene \acs{NWM} befindet sich für\\
        $t-t_s>>\underset{i}{\operatorname{max}}
          \left\{-\frac{1}{\Re{\cn{A}_i}}\right\}$ näherungsweise im EZ.
          Da $v(t)$ rein harmonisch ist, sogar im \acs{HEZ}!
    \end{itemize}\medskip

  % TODO: Kasten unten rechts fehlt!
  \begin{minipage}{0.5\textwidth}
    $\cn{A}_i\colon$ Natürliche Frequenzen
    \[e^{\cn{A}_i\cdot t} = \underset{
        \begin{array}{c}
          \scriptstyle\text{klingt ab}\\
          \scriptstyle\text{für }t\to\infty,\\
          \scriptstyle\text{falls }\sigma<0
        \end{array}}{
      e^{\sigma\cdot t}} \cdot \underset{
        \begin{array}{c}
          \downarrow\\
          |e^{j\omega t}|=1
        \end{array}}
    {e^{j\omega t}}\]
  \end{minipage}
  \begin{minipage}{0.5\textwidth}
    \[\Rightarrow \frac{1}{\sigma}\mathrel{\widehat{=}} \text{Zeitkonstante}\]
    
    $t_m$: Ab diesem Zeitpunkt sind alle $e(t)$ harmonisch.
  \end{minipage}

\end{enumerate}


 % HEZ
% \chapter{10.12.2010}

\chapter{Die (einseitige) Laplacetransformation}
\Index{Laplace-Transformation}

\begin{definition}
  $f(t)$ sei eine Funktion $f\colon \real_t\to\complex$, und
  $\cn{s}\in\complex$.
  
  Existiert
  \[\int\limits_0^{\infty}|f(t)|\cdot e^{-\Re{\cn{s}}t}\diff t < 0\]
  so heißt $f(t)$ (absolut) $\mathfrak L$-Transformierbar.
  
\end{definition}

  Schreibweise: $\cn{T}(\cn{s}) = \LT{f(t)}(\cn{s})
    := \int\limits_0^\infty f(t)\cdot e^{-\cn{s}t}\diff t$
  
  Die Transformation ist auch für $f\colon \real\to\mathbb{C}$
  gültig und eindeutig,
  wenn man \[f(t)\equiv0 \text{ für } t<0\] vorraussetzt!
  D.h., $f(t)\stackrel{!}{=}\Theta(t)\cdot \tilde{f}(t)$
  
\section{Laplace-ESB}%TODO: sort, extend
\Index{Ersatzschaltbild!Laplace}
\begin{example}\hfill

  \begin{circuitikz}[scale=0.8]
    \draw (0,0) to [C,v_>=$u_c(t)$,i>^=$i_c(t)$,*-*] (3,0);
  \end{circuitikz},
  $u_c(0^+) \stackrel{!}{=} U_{C,0}$, wende auf die ZGL die Laplacetrafo an:
  \begin{align*}
    \LT{i_c(t)}(\cn{s})
      &=\int\limits_0^\infty C\cdot \frac{\partial}{\partial t} u_c(t)
        \cdot e^{-\cn{s}t} \diff t  \\
    \intertext{\centering (Anwendung partielle Integration:
      $\int u'\cdot v = \int( u\cdot v)'- \int u\cdot v'$ ) }
      &=\int\limits_0^\infty \left(C\cdot u_C(t)\cdot e^{-\cn{s} t}\right)'
        - C\cdot u_C(t)\cdot (-\cn{s})\cdot e^{-\cn{s}t}\diff t \\
      &=\left[C\cdot u_C(t)\cdot e^{-\cn{s} t}\right]_0^\infty +\cn{s}C
        \cdot\underbrace{\int\limits_0^\infty u_C(t)\cdot e^{-\cn{s}t}\diff t}_{
          \cn{U}_C(\cn{s})}  \\
      &=-C\cdot u_C(0^+)\cdot 1+\cn{s}C\cdot \cn{U}_C(\cn{s})
  \end{align*}
  Das kann man auch als "`NWM"' aufschreiben (Laplace-ESB):
  
  \begin{minipage}{0.5\textwidth}
    \centering
    \begin{circuitikz}[scale=1.0]
      \draw (0,0)
        to [short, o-, i_<=$\cn{I}_C(\cn{s})$] (1,0)
        to [R=$\cn{s}C$,*-*, v_<=$\cn{U}_C(\cn{s})$] (3,0)
        to [short, -o] (4,0)
        (3,0) -- (3,1.5)
        to [I, i_<=$C\cdot u_c(0^+)$] (1,1.5)
        -- (1,0)
    ;\end{circuitikz}
  \end{minipage}
  \begin{minipage}{0.5\textwidth}
    \[\Rightarrow\quad
      \cn{I}_C(\cn{s}) = \cn{s}C\cdot \cn{U}_C(\cn{s})-C\cdot u_C(0^+)\]
  \end{minipage}
\end{example}

Für die in WuN wichtigen Funktionen findet man die $\mathfrak{L}$-Transformierte
in der Tabelle (siehe FS (Formelsammlung)).

\paragraph{Definition} (wichtig!): Eine ÜF ist definiert durch
\Index{Übertragungsfunktion}
  \[\left.\cn{H}(\cn{s}) := 
    \frac{\cn{A}(\cn{s})}{\cn{E}(\cn{s})}\right|_{\begin{array}{l}
      \text{\footnotesize Alle $AW\stackrel{!}{=}0$}\\
      \text{\footnotesize Alle anderen festen Quellen $\stackrel{!}{=}0$}
    \end{array}}
    \qquad
    \begin{array}{lp{3cm}}
      \cn{A}(\cn{s}): &\text{Ausgang}  \\
      \cn{E}(\cn{s}): &\text{Eingang,}
        \text{häufige eine feste Quelle,}
        \text{kann auch eine beliebige}
        \text{Zweiggöße sein}
    \end{array}\]

  Darf man bei der $\mathfrak{L}$-Trafo auch $\cn{s}=j\omega$ einsetzen
  (NUR bei asympt. stabilen NWM möglich), stimmt 
  $\cn{H}(\cn{s})|_{\cn{s}=j\omega}$ mit dem Frequenzgang überein.
\Index{Frequenzgang}

\paragraph{Stoßantwort}
\Index{Stoßantwort}
  Die Stoßantwort $a_\delta(t)$ auf eine Quelle $e(t)$ erhält man durch
  \[\cn{H}(\cn{s}) := \frac{\cn{A}(\cn{s})}{\cn{E}(\cn{s})},\quad
    \cn{E}(\cn{s}) = \LT{\delta(t)}(\cn{s}) \underset{\text{Tabelle}}{=} 1\]
  \infobox{Berechnung der Stoßantwort (wichtig!)}{
    \[\begin{array}{lccc}
          &\cn{A}(\cn{s})  &= &\cn{H}(\cn{s})\,\cdot\, 1  \\
        \mathfrak{L}^{-1} &\InversTransformVert & &\InversTransformVert  \\
          &a_\delta(t) &= &\mathfrak{L}^{-1}\{\cn{H}(\cn{s})\}(t)
      \end{array}\]}

\section{Der Laplace-Faltungssatz}
\Index{Laplace-Faltungssatz}
  \[\LT{(f(t')\ast g(t'))(t)} = \cn{F}(\cn{s})\cdot\cn{G}(\cn{s})\]

  \begin{minipage}{0.45\textwidth}
  \infobox{Extrem wichtig!}{\vspace{-1em} %TODO: check why hack is needed!
    \[\begin{array}{lccc}
      &\cn{A}(\cn{s}) &= &\cn{H}(\cn{s}) \cdot \cn{E}(\cn{s})  \\
      \mathfrak{L}^{-1}\{\}(t)  & \InversTransformVert & &\InversTransformVert\\
      &a_{rz}(t) &= &(a_\delta(t') \ast e(t'))(t)
    \end{array}\]}
  \end{minipage}
  \hfill
  \begin{minipage}{0.5\textwidth}
  \paragraph{Nebenbedingungen:}
    \begin{tabularx}{\textwidth}{lX}
      $\cn{H}(\cn{s})$:
        & Alle AW$\stackrel{!}{=}0$, \newline
          alle anderen festen Quellen $\stackrel{!}{=}0$ \\
      $\cn{E}(\cn{s})$:
        & $e(t)\stackrel{!}{=}0\text{ für } t<0$
    \end{tabularx}\medskip

    Diese Bedingungen entsprechen gerade der Antwort aus dem \acs{RZ} bei $t=0$.
  \end{minipage}

Für asymptotisch stabile NWM im eingeschwungenen Zustand (AW abgeklungen)
gilt ebenfalls:

  \begin{minipage}{0.45\textwidth}
  \infobox{Extrem wichtig!}{
    \[a_{ez}(t) = (a_\delta(t') \ast e_\infty(t'))(t)\]}
  \end{minipage}
  \hfill
  \begin{minipage}{0.5\textwidth}
    $e_\infty(t)$ muss für $t<0$ beschränkt sein!
    (Durch $e^{-\frac t\tau}$, $\tau >\underset{i}{\operatorname{max}}
      \left\{\frac{-1}{\Re{\cn{A}_i}}\right\}$)
  \end{minipage}

\paragraph{Anwendung}
  \begin{itemize}[label=$\rightarrow$]
    \item Faltung im Zeitbereich:
      \begin{enumerate}[label=\arabic*.)]
        \item $e(t)$ is ein PWL-Signal (Weltformel!), Faltungsrechenregeln
          verwenden!
        \item $e(t)$ ist nicht L-transformierbar (behandelt man in WuN nicht,
          z.B. $e(t) = V\cdot e^{\frac{t^2}{\tau^2}}\Theta(t)$
      \end{enumerate}
    \item Faltung im Laplace-Bereich:
      \begin{itemize}[label=$\rightarrow$]
        \item Alle anderen Fälle
      \end{itemize}
  \end{itemize}


\section{Beispiel: H07 A1}

\begin{center}
\begin{tikzpicture}[scale=1.2]
  % coords
  \draw[->,>=latex] (-0.5,0) -- (6,0) node [below right] {$t$};
  \draw[->,>=latex] (0,-0.5) -- (0,2) node [left] {$e(t)$};
  % function
  \draw (0,0) -- ++(1,0) -- ++(1,1) -- ++(1,0) -- ++(2,-1);
  % help functions
  \draw[dashed, green] (-0.5,0.05) -- ++(1.5,0) -- ++(45:2.5) node[left] {\scriptsize $m=E_1$};
  \draw[dashed, gray] (2,1.05) -- ++(-45:2.5) node[right] {\scriptsize $m=E_2=-E_1$};
  \draw[dashed, red] (2,1.05) -- ++(3,0);
  \draw[dashed, green] (3,1.05) -- ++(3,-1.5) node[below right] {\scriptsize $m=E_3$};
  \draw[dashed, red] (5,0.05) -- ++(1.5,0);
  \draw[dashed, gray] (5,0.05) -- ++(27:1.5) node[right] {\scriptsize $m=E_4=-E_3$};
  % time
  \draw (1,0.1) -- +(0,-0.2) node [below] {$T_1$};
  \draw (2,0.1) -- +(0,-0.2) node [below] {$T_2$};
  \draw[dashed] (2,0) -- ++(0,1);
  \draw (3,0.1) -- +(0,-0.2) node [below] {$T_3$};
  \draw[dashed] (3,0) -- ++(0,1);
  \draw (5,0.1) -- +(0,-0.2) node [below] {$T_4$};
  % J0
  \draw[dashed] (2,1) -- ++(-2.1,0) node [left] {$J_0$};
\end{tikzpicture}
\end{center}
% 
\begin{align*}
  j(t) &= \;\; \underbrace{\left(\frac{J_0}{T_2-T_1}\right)}_{E_1}
    \cdot(t-T_1) \cdot\Theta(t-T_1)
  + \underbrace{\left(-\frac{J_0}{T_2-T_1}\right)}_{E_2}
    \cdot(t-T_2) \cdot\Theta(t-T_2)\\
  &\;\; + \underbrace{\left(-\frac{J_0}{T_4-T_3}\right)}_{E_3}
    \cdot(t-T_3) \cdot\Theta(t-T_3)
  + \underbrace{\left(\frac{J_0}{T_4-T_3}\right)}_{E_4}
    \cdot(t-T_4) \cdot\Theta(t-T_4) \\
  &= \sum_{k=1}^{4} E_k\cdot \Theta(t-T_k)\cdot(t-t_k)
\end{align*}

\paragraph{Verschiebungssatz}
\[\left(e(t'{\color{blue}-\tau})\ast a_g(t')\right)(t)
  = \left(e(t')\ast a_g(t')\right)(t{\color{blue}-\tau})\]



\section{Beispiel: H08 A1 e)}

\acs{ESB} im Laplace-Bereich (\acs{AW}$\stackrel{!}{=}0$):

\begin{minipage}{0.5\textwidth}
\begin{circuitikz}
  \draw (0,0) 
    -- (0,1)
    to[I=$\cn{J}(\cn{s})$] (0,3)
    -- (0,4)
    -- (2,4)
    to[R=$R$] (4,4)
    to[short,i=$\cn{I}_2(\cn{s})$] (4,0)
    -- (0,0)
    (2,4)
    to[C=$C$,*-] (2,2)
    to[R=$R$,-*] (2,0)
;\end{circuitikz}
\end{minipage}
\begin{minipage}{0.5\textwidth}
\begin{align*}
  \cn{H}(\cn{s})
    &=\frac{\cn{I}_L(\cn{s})}{\cn{J}(\cn{s})} \\
    &=\frac{R+\frac{1}{\cn{s}C}}{\left(R+\frac{1}{\cn{s}C}\right)+R}  \\
    &=\frac{\cn{s}RC+1}{\cn{s}2RC+1}  \\
    &=\frac{RC}{2RC}\cdot\frac{\cn{s}+\frac{1}{RC}}{\cn{s}+\frac{1}{2RC}} \\
    &=\frac{1}{2}\left(\frac{\cn{s}+\frac{1}{2RC}-\frac{1}{2RC}+\frac{1}{RC}}
      {\cn{s}+\frac{1}{2RC}}\right) \\
    &=\frac{1}{2}\left(1+\frac{1}{2RC}\cdot\frac{1}{\cn{s}+\frac{1}{2RC}}\right)
\end{align*}
\end{minipage}

\begin{enumerate}
  \item Antwort aus dem \acs{RZ}? \\
    (oder) Antwort im EZ? \\
    (oder) ``zero-state''-response berechnen?

    Wenn ja: Faltung!

    Hier: kein \acs{AW}, $j(t)=0$ für $t<T_1 \Rightarrow$AW aus dem \acs{RZ}!
  \item
    \colorbox{blue!20}{
    \begin{minipage}{0.6\textwidth}
    \[e(t) =\underbrace{\sum_{k=1}^{n} E_k
          \cdot\Theta(t-T_k)\cdot(t-T_k)}_{
            \text{Knick-Funktionen}}
        +\underbrace{\sum_{k=1}^{n}F_k\Theta(t-T_k)}_{\text{Sprünge} }
    \]
    \end{minipage}}
    \hfill
    \begin{minipage}{0.3\textwidth}
            \footnotesize allgemeine Fkt.-Gleichung eines \acs{PWL}-Signals,
            nur für diese funktioniert die ``Welt-Formel''!
    \end{minipage}
  
  \begin{minipage}{0.3\textwidth}
  \begin{tikzpicture}
    % coords
    \draw[->,>=latex] (-0.5,0) -- (4,0) node [below right] {$t$};
    \draw[->,>=latex] (0,-0.5) -- (0,2) node [left] {$j(t)$};
    % function
    \draw[blue] (0,0) -- (1,0);
    \draw[dashed,blue] (1,0) -- (1,1);
    \draw[blue] (1,1) -- (2,1) -- (3,0) -- (4,0);
    % time
    \draw (1,0.1) -- +(0,-0.2) node [below] {$T_1$};
    \draw (2,0.1) -- +(0,-0.2) node [below] {$2T_1$};
    \draw (3,0.1) -- +(0,-0.2) node [below] {$3T_1$};
    % J0
    \draw[dashed] (0.1,1) -- ++(-0.2,0) node [left] {$J_0$};
  \end{tikzpicture}
  \end{minipage}%
  \begin{minipage}{0.7\textwidth}
    \begin{align*}
      e(t) &=\underbrace{(J_0-0)\cdot\Theta(t-T_1)}_{
          E_1:=0,\, F_1:=J_0}\\
        &\;\;\;+\underbrace{\left(-0+\frac{0-J_0}{3T_1-2T_1}\right)
          \cdot\Theta(t-2T_1)\cdot(t-2T_1)}_{
            E_2:=-\frac{J_0}{T_1},\, F_2:=0,\, T_2:=2T_1}\\
        &\;\;\;+\underbrace{\left(-\frac{0-J_0}{3T_1-2T_1}+0\right)
          \cdot\Theta(t-3T_1)\cdot(t-3T_1)}_{
            E_3:=-\frac{J_0}{T_1},\, F_3:=0,\, T_3:=3T_1}  \\
      &=\sum_{k=1}^{3} E_k\Theta(t-T_k)\cdot(t-T_K)
        +\sum_{k=1}^{3} F_k\Theta(t-T_k)
    \end{align*}
  \end{minipage}


  \item Bestimmung der Stoßantwort mittels inverser Laplace-Transformation
    \[a_\delta(t) = \mathfrak{L}^{-1}\{\cn{H}(\cn{s})\}(t)\]
    % 
    \[\begin{array}{lccll}
      &\cn{H}(\cn{s})
        &= &\frac{1}{2}&+\dfrac{1}{4RC}\cdot\dfrac{1}{\cn{s}+\frac{1}{2RC}}\\
      \mathfrak{L}^{-1} &\InversTransformVert & 
        &\InversTransformVert&\qquad\qquad\InversTransformVert  \\
      &a_\delta(t) &= &\frac{1}{2}\delta(t) 
        &+ \dfrac{1}{4RC}\cdot\left[\Theta(t)\cdot e^{-\frac{t}{2RC}}\right]
        \quad\leftarrow\text{Distributionsklammern}
    \end{array}\]

\end{enumerate}


\chapter{Faltung im Zeitbereich ("`Weltformel"')}
\Index{Faltung}\Index{Weltformel}

\begin{skills}
  \item Polynomdivision
  \item Partialbruchzerlegung
\end{skills}


\begin{enumerate}
  \item Voraussetzung dafür:
    \begin{enumerate}
      \item Darf man mit Faltung rechnen? Beweisen!
        % rz/ez!?
      \item Ist $e(t)$ ein \acs{PWL}-Signal?
      \item Hat $\cn H(\cn s)$ nur einfache, reelle Pole? (Nicht notwendig)
    \end{enumerate}

  \item Ist c) erfüllt: Berechne $a_\delta(t) = \mathrm{L}^{-1}\{\cn H(\cn s)\}$
    mit Poly-Div. / PBZ

  \item Definiere Hilfsfunktion $g_1(t)$, $g_2(t)$:
    \begin{description}
      \item[Ist c) erfüllt:]
        \begin{align*}
          g_1(t) &:= \left(\Theta(t')\ast a_\delta(t')\right)(t)\\
            & \text{ mit } (\Theta(t')\ast\Theta(t')\cdot e^{-at})(t)
            = \Theta(t)\cdot\frac{1}{a} \left(1-e^{-at}\right)\\
          g_2(t) &:= \left(\Theta(t')\cdot t'\ast a_\delta(t')\right)(t)\\
            & \text{ mit } (\Theta(t')\cdot t'\ast\Theta(t')\cdot e^{-at})(t)
            = \Theta(t)\cdot\frac{1}{a^2} \left(at-(1- e^{-at})\right)
        \end{align*}
        \klausurhinweis{Die benötigten Hilfsfaltungen müssen (inkl. Herleitung) einmal hingeschrieben werden!}
      \item[sonst:]
        \[
        \begin{aligned}
          g_1(t) &:= \mathfrak{L}^{-1}\left\{\cn H(\cn s)
            \cdot \frac{1}{\cn s}\right\}(t)\\
          g_2(t) &:= \mathfrak{L}^{-1}\left\{\cn H(\cn s)
            \cdot \frac{1}{\cn s^2}\right\}(t)
        \end{aligned}
        \qquad
        \text{ mit Polynomdivision\,/\,PBZ}
        \]
    \end{description}
    \infohinweis{$g_1(t) = a_\Theta(t)$ nennt man auch \emph{Sprungantwort!}}

  \item Berechne $a(t)$ per Superposition mit der Faltung:
    \[
      a(t) = (e(t')\ast a_\delta(t'))(t)
        \stackrel{\text{%
          \begin{minipage}{9em}Linearität +\\Verschiebungssatz\end{minipage}
        }}{=}
      \sum_{k=1}^n E_{1k}\cdot g_1(t-T_k) + E_{2k}\cdot g_2(t-T_k)
    \]
\end{enumerate}

% \chapter{14.01.2011}

\chapter{Transistorschaltungen}
\Index{Transistorschaltung}

\paragraph{Allgemeine Vorgehensweise}

\begin{enumerate}
	\item \acs{AP} ausrechnen (immer mit einem DC-\acs{ESB}!)
	\item Kleinsignalparameter aus dem Kennlinienfeld ablesen
	\item \acs{KSig}-\acs{ESB} zeichnen
\end{enumerate}
\Index{Arbeitspunkt}
\Index{Kleinsignalparameter}
\Index{Kleinsignal-Ersatzschaltbild}

\section{AP ausrechnen}
\Index{Arbeitspunkt}

Alle Zweiggrößen des \acs{NWM} sind Gleichgrößen
	\begin{itemize}
		\item $C \rightarrow$ Leerlauf
		\item $L \rightarrow$ Kurzschluss
		\item Gesteuerte
			\begin{tabular}{l}
				SPQ\\ STQ
			\end{tabular}, die durch Ableitungen gesteuert werden, werden durch
			\begin{tabular}{l}
			Kurzschluss\\ Leerlauf
			\end{tabular} ersetzt.

\begin{description}\item[Beispiel:]\hfill\\[-2ex]
\begin{center}
\begin{circuitikz}[scale=1.2]
\draw
	(0,0) to[V=$v(t)\eq\alpha\cdot\frac{\partial}{\partial t}u_R(t)$,*-*] (0,-2)
	
	(5,0) to[R,v^=$U_R(t)  \underset{\text{DC-Fall}}{\eq} U_R$,*-*] (5,-2)
	
	(0,-2.5) node[anchor=west] {$\widehat{=}\; v(t) = \alpha \cdot 0 = 0$}
	
	(0,-3) to[short,*-*] (0,-4);
\end{circuitikz}
\end{center}
\end{description}

\item Kleinsignalquellen $\left(v_{KS}(t), j_{KS}(t)\right)$ werden zu 0
gesetzt.
	\begin{figure}[h]
\begin{center}
\begin{circuitikz}[scale=1.2]
\draw
	(0,0) to[V=$v(t) \eq V_{DC} + v_{in,KS}(t)$,*-*] (0,-2)
	
	(6,0) to[V=$V_{DC}$,*-*] (6,-2)
	(4,-1) node[anchor=west] {$\underset{\text{DC-Fall}}{\longrightarrow}$};
\end{circuitikz}
\end{center}
\end{figure}
  \renewcommand{\labelitemi}{$\Rightarrow$}
  \item Nur DC-Quellen, gesteuerte Quellen ohne Ableitung,
  Widerstände und Transistoren im DC-ESB!
  (NWM im zeitunabhängigen eingeschwungenen Zustand)
\end{itemize}


% \newpage

	\begin{figure}[h]
\caption*{DC-ESB}
\begin{center}
\begin{circuitikz}[scale=0.8]
\draw
	(-1.5,1.5) node[pigfete,anchor=G] (tr) {}
	(tr.G) node[anchor=south] {G}
	(tr.D) node[anchor=east] {D}
	(tr.S) node[anchor=west] {S}
	
  % NW 2
  (-6,-1) |- (-4,1) |- (-6,-1)
  (-5,0) node {NW2}
  % NW 1
  (3, -1) |- (5, 1) |- (3,-1)
  (4,0) node {NW1}

  (-5,-1.5) -- (-5,-1)
  (-5,1) -- (-5,1.5)
    -- (-2,1.5) to[short,o-] (tr.G)
  (-0.25, -1.5)
    to[short, -o] (1.5,-1.5) -- (4, -1.5) -- (4, -1)
% 	(4,1) -- (4,2.75) -- (1.5, 2.75) to[short, o-,i_=$I_D$] (tr.D)
  (tr.D) to [short,-o,i_<=$I_D$] ($(tr.D)+(2,0)$) -| (4,1)
  (tr.S) to[short, -o] (-0.25, -1.5) -- (-5,-1.5)

	(1.5, 2.75) to[open,v^>=$U_{DS}$] (1.5, -1.5)
	(-2,1.5) to[open,v>=$U_{GS}$] (0, -1.5)
	;
\end{circuitikz}
\end{center}

\end{figure}


Bei MOS-Transistoren: $I_{G,DC} = 0$

\paragraph[AG]{\acs{AG} 1:}


Bei einfachen NWM: $I_D = f(U_{DS})$

\begin{itemize}
	\item Maschengleichung $M$ aufstellen und nach $I_D$ auflösen
	\item NW1 als \acs{ESPQ} darstellen
	\subitem $\to$ \acs{LL}-Spannung von NW1 bzgl: $D$ und $S$
\end{itemize}


\paragraph[AG]{\acs{AG} 2:}

\begin{itemize}
	\item Linke Seite genauso mit $I_G = 0$
	\item NW2 als \acs{ESPQ} bzgl. $G$ und $S$
\end{itemize}

% \begin{figure}
% \caption*{Grausames bild}
\begin{center}
\begin{tikzpicture}[scale=1.2]
  % Koordinatensystem
  \draw[->,>=latex] (0,0) node [below=0.4] {(NMOS)}
    -- (7,0) node [below right] {$U_{DS}$};
  \draw[->,>=latex] (0,0) -- (0,4) node [above left] {$I_{DS}$};
  \draw[->,>=latex] (7.5,3.0) -- node [right] {$U_{GS}$} (7.5,4.0);
  % Kennlinie
  \draw (0,0) to[out=60,in=190] (2.5,2.6) -- +(10:4);
  \draw (0,0) to[out=50,in=190] (2.5,2.0) -- +(10:4);
  \draw (0,0) to[out=40,in=190] (2.5,1.4) -- +(10:4) node [right] {$U_{GS}=1V$};
  \draw (0,0) to[out=30,in=190] (2.5,0.8) -- +(10:4);
  % AGs
  \draw[blue] (-0.1,3) node [left] {$\frac{V_{LL,NW1}}{R_{,NW1}}$}
    -- (0,3)
    -- (6,0)
    -- (6,-0.1) node [below] {$V_{LL,NW1}$};
  % Kreuze vertikal
  \node[cross out,draw,blue]  at ($(2.5,0.8)+(10:0.52)$) {};
  \node[cross out,draw,blue]  at ($(2.5,1.4)+(10:0.52)$) {};
  \node[cross out,draw,blue]  at ($(2.5,2.0)+(10:0.52)$) {};
  \node[cross out,draw,blue]  at ($(2.5,2.6)+(10:0.52)$) {};
  \draw[blue] ($(2.5,2.6)+(10:0.52)+(90:0.5)$) -- +(-90:3.5);
  % Tangente
  \draw[blue] ($(2.5,1.4)+(10:0.52)+(10:-3.2)$) node [left] {Tangente}
    -- +(10.1:6);
\end{tikzpicture}
\end{center}
% \end{figure}

\section{KS-Parameter}

\begin{align*}
  R_D^{-1}
    &= \left.\left(\frac{\partial I_D}{\partial U_D}\right)(AP)\right|_{
      U_G=\text{konst}=U_{G,DC}}
    &&\approx \frac{\triangle I_{DS}}{\triangle U_{DS}}
      \quad \text{der Tangente}\\
  g
    &= \left.\left(\frac{\partial I_D}{\partial U_G}\right)(AP)\right|_{
      U_D=\text{konst}=U_{D,DC}}
    &&\approx \frac{\triangle I_{DS}}{\triangle U_G}
      \quad \text{der Sekante} = \frac{I_{D,x1}-I_{D,x2}}{U_{G,x1}-U_{G,x2}}
\end{align*}

Aus AG2: $U_{GS}$ direkt ablesen (in den meisten Fällen) z.\,B. $U_{GS} =
1\text{V}$

Allgemein: $I_{DS}= f(U_{GS}) \Rightarrow$ Kennlinie Punktweise einzeichnen. Der Schnittpunkt beider Kennlinien ist der AP!

%%% HIER FEHLT NOCH EINIGES!

\paragraph{Näherung: Schaltzwang}\hfill

\begin{minipage}{0.5\textwidth}\centering
\begin{circuitikz}
  \draw node[nmos, anchor=D] (pmos) at (0,0) {}
    (pmos.D) node [right] {D}
    (pmos.G) node [below] {G}
    (pmos.S) node [right] {S}
    (pmos.D) to[short,i<=$I_D$] (0,0.75) to[short,*-o] (0,1.5)
    (pmos.S) to[short,-o] (0,-2)
    (0,0.75) -| ($(pmos.G)+(-0.6,0)$) to[short,i=$I_G$] (pmos.G)
;\end{circuitikz}
\end{minipage}
\begin{minipage}{0.5\textwidth}
\[U_{DS}\stackrel{!}{=}U_{GS}\]
Man kann sich nur auf dieser Kennlinie bewegen. $\rightarrow$ Einzeichnen!
\end{minipage}


%%% HIER FEHLT NOCH EINIGES!


\begin{circuitikz}
  \draw (0,0) node [circle,draw,inner sep=0.5mm,left=0.3] {S}
    to[R=$R_{ges}$,o-o] (3,0)
      node [circle,draw,inner sep=0.5mm,above right=0.3] {G}
      node [circle,draw,inner sep=0.5mm,below right=0.3] {D}
;\end{circuitikz}


\paragraph{"`Normalfall"'}\hfill

\begin{circuitikz}
  \draw (0,0)
      to[open,o-o] (0,2)
        node [draw,circle,inner sep=0.5mm,left=0.3] {G}
    (0,0) to[short,o-o] (5,0)
        node [circle,draw,inner sep=0.5mm,right=0.3] {S}
    (3,2) to[I_=$g\cdot U_{GS}(t)$,*-] (3,0)
    (4,2) to[R=$R_D$,*-*] (4,0)
    (3,2) to[short,-o] (5,2) node [circle,draw,inner sep=0.5mm,right=0.3] {D}
;\end{circuitikz}

\section{Kleinsignal-ESB}

\begin{enumerate}[label=\alph*)]
  \item KS-ESB der Schaltung\\
    $\rightarrow$ Transistor-KS-ESB einsetzen\\
    $\rightarrow$ DC-Quellen $\stackrel{!}{=}0$ setzen.
  \item fertig!!
\end{enumerate}

\paragraph{Anmerkung}

H04, F05, H05 sind Einsteigeraufgaben.

Teil d) kann erst beantwortet werden, wenn e) gelöst wurde



% \chapter{28.01.2011}

\chapter{Vierpole/Zweitore}
\Index{Vierpol}\Index{Zweitor}

\begin{minipage}{0.6\textwidth}\centering
\begin{circuitikz}
  \draw (0,0)
    -- (2,0)
    -- (2,3)
    -- (0,3)
    -- (0,0)
    (0,0.5) to[short,-o,i_>=$\cn{I'}_1$] (-1,0.5) node [below] {Tor 1}
    (0,2.5) to[short,-o,i_<=$\cn{I}_1$] (-1,2.5)
    (2,0.5) to[short,-o,i^>=$\cn{I'}_2$] (3,0.5) node [below] {Tor 2}
    (2,2.5) to[short,-o,i^<=$\cn{I}_2$] (3,2.5)
    (-1,0.5) to[open,v^=$\cn{U}_1$] (-1,2.5)
    (3,0.5) to[open,v_=$\cn{U}_2$] (3,2.5)
;\end{circuitikz}
\end{minipage}
\begin{minipage}{0.4\textwidth}
Definition eines Zweitors\\
(Richtung $\cn{I}_{1,2}$, $\cn{U}_{1,2}$!)
\end{minipage}\medskip

\paragraph{Matrixdarstellung:}
\[\begin{array}{clccc}
    \begin{pmatrix}\cn{A}_1\\\cn{A}_2\end{pmatrix}
    &= \begin{pmatrix}
        \cn{M}_{11} & \cn{M}_{12} \\
        \cn{M}_{21} & \cn{M}_{22}
      \end{pmatrix}
    &\cdot\begin{pmatrix}
          \cn{E}_1  \\  \cn{E}_2
         \end{pmatrix}
    &+&\begin{pmatrix}
        \cn{Q}_1  \\ \cn{Q}_2
      \end{pmatrix} \\
  \uparrow  && \uparrow &&\uparrow\\
  \begin{minipage}{3.5cm}\small\raggedright
    \text{Ausgangsvektor,}
    \begin{description}
      \item[$\cn{A}_1, \cn{A}_2\colon$]
      Ausgangs\-zeiger aus
      \{$\cn{I}_1$, $\cn{I}_2$, $\cn{U}_1$, $\cn{U}_2$\}
    \end{description}
  \end{minipage}
  &&\begin{minipage}{3cm}\small\raggedright
      Eingangsvektor,
      $\cn{E}_1$, $\cn{E}_2$ sind die beiden
      anderen Zeiger aus
      \{$\cn{I}_1$, $\cn{I}_2$, $\cn{U}_1$, $\cn{U}_2$\}
   \end{minipage}
   &&\begin{minipage}{1.5cm}\small\raggedright
      Vektor der festen Quellen
    \end{minipage}  \\
  \downarrow  &&  \downarrow \\
  \text{\small Rechnet man aus}
  &&\text{\small Gibt man vor}
  \end{array}\]

\paragraph{Vorgehensweise:}
\begin{enumerate}[label=\arabic*)]
  \item {\boldmath$\cn{Q}_1$, $\cn{Q}_2$} bestimmen:
    Setze $\cn{E}_1 = \cn{E}_2 \stackrel{!}{=} 0
    \Rightarrow \begin{pmatrix}\cn{A}_1\\\cn{A}_2\end{pmatrix}
    =\begin{pmatrix}\cn{Q}_1\\\cn{Q}_2\end{pmatrix}$
    \begin{example}
      \begin{tabular}[t]{ll}
        $\cn{E}_1$ ist eine Spannung: & KS an Tor 1!  \\
        $\cn{E}_2$ ist ein Strom:     & LL an Tor 2!  \\
      \end{tabular}
    \end{example}

  \item {\boldmath$\cn{M}_{11}$, $\cn{M}_{21}$} bestimmen:
    Alle festen Quellen $\stackrel{!}{=} 0 \Rightarrow \cn{Q}_1, \cn{Q}_2 = 0$
    
    $\cn{E}_2 \stackrel{!}{=} 0$
    \[\Rightarrow\begin{pmatrix}\cn{A}_1\\\cn{A}_2\end{pmatrix}
      = \begin{pmatrix}
          \cn{M}_{11}\cdot\cn{E}_1\\
          \cn{M}_{21}\cdot\cn{E}_1
        \end{pmatrix}
      \Rightarrow\begin{aligned}
        \cn{M}_{11} = \left.\frac{\cn{A}_1}{\cn{E}_1}\right|_{
          \cn{Q}_1=\cn{Q}_2=\cn{E}_2\stackrel{!}{=}0}\\
        \cn{M}_{21} = \left.\frac{\cn{A}_2}{\cn{E}_1}\right|_{
          \cn{Q}_1=\cn{Q}_2=\cn{E}_2\stackrel{!}{=}0}
      \end{aligned}\]

%     $\rightarrow \cn{E}_1\colon$
%     \begin{tabular}{@{}ll@{}}Stromquelle\\Spannungsquelle\end{tabular}
%     einzeichnen
% 
%     $\cn{E}_2$, alle festen
%     \begin{tabular}{@{}l@{}}Spannungsquellen\\Stromquellen\end{tabular}
%     werden zu
%     \begin{tabular}{@{}l@{}}KS\\LL\end{tabular}
    
    \begin{tabular}{l@{\,\,:\quad}l}
      $\cn{E}_1$ & \begin{tabular}{@{}ll@{}}Stromquelle\\Spannungsquelle\end{tabular}
      einzeichnen\\
      $\cn{E}_2$, alle festen
    \begin{tabular}{@{}l@{}}SPQ\\STQ\end{tabular} &
      werden zu
    \begin{tabular}{@{}l@{}}KS\\LL\end{tabular}
    \end{tabular}


  \item {\boldmath$\cn{M}_{12}$, $\cn{M}_{22}$} bestimmen:
    Alle festen Quellen $\stackrel{!}{=} 0 \Rightarrow \cn{Q}_1$, $\cn{Q}_2=0$
    
    $\cn{E}_1 \stackrel{!}{=} 0$
    \[\Rightarrow\begin{pmatrix}\cn{A}_1\\\cn{A}_2\end{pmatrix}
      = \begin{pmatrix}
          \cn{M}_{12}\cdot\cn{E}_2\\
          \cn{M}_{22}\cdot\cn{E}_2
        \end{pmatrix}
      \Rightarrow\begin{aligned}
          \cn{M}_{12} = \left.\frac{\cn{A}_1}{\cn{E}_2}\right|_{
            \cn{Q}_1=\cn{Q}_2=\cn{E}_1\stackrel{!}{=}0} \\
          \cn{M}_{22} = \left.\frac{\cn{A}_2}{\cn{E}_2}\right|_{
            \cn{Q}_1=\cn{Q}_2=\cn{E}_1\stackrel{!}{=}0}
        \end{aligned}
    \]
% 
%     $\cn{E}_1$, alle festen Quellen werden zu
%       \begin{tabular}{l}
%         KS\\LL
%       \end{tabular}
% 
%     $\rightarrow \cn{E}_2\colon$
%       \begin{tabular}{@{}ll@{}}Stromquelle\\Spannungsquelle\end{tabular} einzeichnen
      
    \begin{tabular}{l@{\,\,:\quad}l}
      $\cn{E}_1$, alle festen
      \begin{tabular}{@{}l@{}}SPQ\\STQ\end{tabular} &
      werden zu
      \begin{tabular}{@{}l@{}}KS\\LL\end{tabular}\\
      $\cn{E}_2$ & \begin{tabular}{@{}ll@{}}Stromquelle\\Spannungsquelle\end{tabular}
      einzeichnen
    \end{tabular}
\end{enumerate}


% \chapter{11.02.2011}

\chapter{Quellenverschiebung}

\emph{Wann muss man Quellen verschieben?}

\begin{minipage}{0.5\textwidth}
\quad\begin{tabular}{@{}ll}
  Maschen-$\cn{Z}$-Verfahren:       & STQ \\
  Schnittmengen-$\cn{Y}$-Verfahren: & SPQ \\
  Knoten-$\cn{Y}$-Verfahren:        & SPQ
\end{tabular}
\end{minipage}
\begin{minipage}{0.5\textwidth}
  aber \emph{nur}, wenn sie allein in einem Zweig vorkommen
  (d.h. \emph{ideale} Quellen)! Sonst: Quellensubstitution
\end{minipage}

\emph{Wie verschiebt man?}

\begin{minipage}{0.5\textwidth}
\quad\begin{tabular}{@{}l}SPQ\\STQ\end{tabular}verschiebt
man in \begin{tabular}{@{}l@{}}Schnittmengen\\Maschen\end{tabular}\,!
\end{minipage}
\begin{minipage}{0.5\textwidth}
\begin{tabular}{@{$\to$\,}l}
  \emph{Nur} die Zweigspannungen ändern sich!\\
  \emph{Nur} die Zweigströme ändern sich!\\
\end{tabular}
\end{minipage}


\emph{Welche Gleichungen muss man aufstellen?}

\begin{description}
  \item[Transforamationsgleichungen:]
    Neue Zweiggrößen (gestrichene Größen) durch die alten Zweiggrößen
    oder/und feste Quellen ausdrücken
    
    $\Rightarrow$ 1 Gl. pro Zweig, in den man eine Quelle hinein verschiebt.
  
  \item[Rücktransformationsgleichungen:]
    Alte Zweiggrößen durch neue Zweiggrößen und/oder feste Quellen ausdrücken,
    $\Rightarrow$ Trafo-Gleichungen umstellen!
    
    $\Rightarrow$ 1 Gl. pro Zweig, in den man eine Quelle hinein verschiebt.
    
  \item[Rückgewinnungsgleichungen:]
    Zweigspannung und -strom der Quelle berechnen, die man verschiebt.
    
    \begin{tabularx}{\textwidth}{lX}
      Bei SPQ: & Zweigspannung $\stackrel{!}{=}$ Quellspannung,\newline
        Zweigstrom durch SGL in die man verschiebt darstellen!\\
      Bei STQ: & Zweigstrom $\stackrel{!}{=}$ Quellstrom,\newline
        Zweigspannung durch MGL in die man verschiebt darstellen! 
    \end{tabularx}
    $\Rightarrow$ 2 Gleichungen
    
    \klausurhinweis{%
      Die ist notwendig, wenn in der Aufgabe steht:
        \glqq Geben Sie die vollständigen Quellentransformationsbeziehungen an\grqq.
      Sonst müssen nur die Gleichungen aufgeschrieben werden, die man später braucht
      (Steuergrößen/zu berechnende Größen).
    }
\end{description}
\paragraph{Bemerkungen}
\begin{itemize}
  \item
Steuerungen transformieren mit Rücktrafo-/Rückgewinnungs-Gl., falls nötig

Evtl.: Steuerungstransformation erst einsetzen, wenn man die Matrix schon
aufgestellt hat!

%   \item
% Man muss \begin{tabular}{l}SPQ\\STQ\end{tabular} beim
% \begin{tabular}{l}Schnittmengen-\\Maschenimpedanz-\end{tabular}Verfahren
% verschieben, wenn sie in einem Zweig allein ($\rightarrow$ ideal) vorkommen:
% %%%
% bzw.
% %%%
% , gesteuert oder ungesteuert

%   \item
% Alle Trafo-/Rücktrafo, Rückgewinnungsgl: "`Stellen Sie die vollst.
% Quellentrafos auf!"'
\end{itemize}


\section{SPQ-Verschiebung}

\begin{minipage}{0.5\textwidth}
\begin{circuitikz}
  \draw(0,0.5)
      node[draw,circle,inner sep=0.5mm,blue] [above left = 0.3] {$1$}
    to[short,o-,i_=$\cn{I}_{V}(\cn{s})$] (0,0)
    to[V,v=$\cn{V}(\cn{s})$] (0,-2)
    to[short,-*] (0,-2.5)
      node[draw,circle,inner sep=0.5mm,blue] [left = 0.7] {$2$}
    to[short] (0,-3)
    to[R=$R_2$] (0,-5)
    to[short,-o,i_=$\cn{I}_{R_2}(\cn{s})$] (0,-5.5)
      node[draw,circle,inner sep=0.5mm,blue] [below left = 0.3] {$3$}
    (0,-2.5)
    to[short] (0.5,-2.5)
    to[R,l_=$R_1$] (2.5,-2.5)
    to[short,-o,i_=$\cn{I}_{R_1}(\cn{s})$] (3.0,-2.5)
      node[draw,circle,inner sep=0.5mm,blue] [above right = 0.3] {$4$}
    (-0.3,0) to[open,v_>=$\cn{U}_{V}(\cn{s})$] (-0.3,-2.5)
    (-0.3,-2.5) to[open,v_>=$\cn{U}_{R_2}(\cn{s})$] (-0.3,-5.5)
    (0,-2.2) to[open,v^>=$\cn{U}_{R_1}(\cn{s})$] (3.0,-2.2)
    ;
  \draw
    (0,-2.5)[dashed,blue] circle (0.5);
  \draw
    ($(0,-2.5)+(50:0.5)$) [dashed,blue,<-,>=latex] -- +(50:0.5);

  \begin{scope}[yshift=-7.5cm]
  \draw(0,0) 
      node[draw,circle,inner sep=0.5mm,blue] [above = 0.7] {$2$}
      node[draw,circle,inner sep=0.5mm,blue] [left = 0.7] {$1$}
    to[short,*-] (0,-0.5)
    to[V_=$\cn{V}(\cn{s})$] (0,-2.5)
    to[R=$R_2$] (0,-4.5)
    to[short,-o,i_=$\cn{I}_{R_2}(\cn{s})$] (0,-5)
      node[draw,circle,inner sep=0.5mm,blue] [below left = 0.3] {$3$}
    (0,0)
    to[short] (0.5,0)
    to[V_=$\cn{V}(\cn{s})$] (2.5,0)
    to[R=$R_1$] (4.5,0)
    to[short,-o,i_=$\cn{I}_{R_1}(\cn{s})$] (5,0)
      node[draw,circle,inner sep=0.5mm,blue] [above right = 0.3] {$4$}
    (0,0.7)
    to[open,v^>=$\cn{U}_{R_1}'(\cn{s})$] (5,0.7)
    (0.7,0)
    to[open,v^>=$\cn{U}_{R_2}'(\cn{s})$] (0.7,-5);
  \draw
    (0,0)[dashed,blue] circle (0.5);
  \draw
    ($(0,0)+(50:0.5)$) [dashed,blue,<-,>=latex] -- +(50:0.5);
  \end{scope}
\end{circuitikz}
\end{minipage}
\begin{minipage}{0.5\textwidth}
\begin{enumerate}[label=\arabic*.)]
  \item
    Schnittmenge neu hinzeichnen, alle Zweiggrößen einzeichnen,
    ggf. selbst die Richtung definieren
  \item
    Schnittmengenrichtung so wählen, dass sie in Richtung der SPQ zeigt (Hilfsmittel)
  \item
    SPQ verschieben:
    \begin{itemize}
      \item Alte SPQ $\to$ Kurzschluss\\
        $\to$ Betreffende Knoten fallen zusammen
      \item Neue SPQ in alle anderen Zweige der Schnittmenge so einzeichnen, dass
        sie der Schnittmengenrichtung \emph{entgegen} gerichtet sind
      \item Neue Zweigspannungen einführen.
        Die Zweigströme ändern sich \emph{nicht}!
    \end{itemize}
\end{enumerate}
\end{minipage}

\paragraph{Trafo-Gl.:}
\[\begin{pmatrix}
    \cn{U}_{R_1}' \\
    \cn{U}_{R_2}'
  \end{pmatrix}(\cn{s})
  = \begin{pmatrix}
      \cn{U}_{R_1} \\
      \cn{U}_{R_2}
    \end{pmatrix}(\cn{s})
    +\cn{V}(\cn{s})
    \begin{pmatrix}1\\1\end{pmatrix}
\]

\paragraph{Rück-Trafo:}
\[\begin{pmatrix}
    \cn{U}_{R_1} \\
    \cn{U}_{R_2}
  \end{pmatrix}(\cn{s})
  = \begin{pmatrix}
      \cn{U}_{R_1}' \\
      \cn{U}_{R_2}'
    \end{pmatrix}(\cn{s})
    -\cn{V}(\cn{s})
    \begin{pmatrix}1\\1\end{pmatrix}\]

\paragraph{Rückgewinnungs-Gl.:}
\[\begin{aligned}
  \cn{U}_V(\cn{s})  &= \cn{V}(\cn{s}) \\
  \cn{I}_V(\cn{s})  &= \cn{I}_{R_1}(\cn{s}) + \cn{I}_{R_2}(\cn{s})
\end{aligned}\]


\section{STQ-Verschiebung}

\begin{minipage}{0.5\textwidth}
\begin{circuitikz}
  \draw(0,0) 
      node[draw,circle,inner sep=0.5mm,blue] [above left = 0.3] {$1$}
    to[short,o-] (0.5,0)
    to[R,l_=$\cn Z_1$] (2.5,0)
    to[short,-*,i=$\cn I_{Z_1}(\cn{s})$] (3,0)
      node[draw,circle,inner sep=0.5mm,blue] [above right = 0.3] {$2$}
    to[short] (3,-0.5)
    to[R,l_=$\cn Z_2$] (3,-2.5)
    to[short,i<=$\cn I_{Z_2}(\cn{s})$] (3,-3.0)
    to[short,-*] (0,-3.0)
      node[draw,circle,inner sep=0.5mm,blue] [below left = 0.3] {$3$}
    to[short,i=$\cn I_J(\cn{s})$] (0,-2.5)
    to[I_=$\cn J$] (0,-0.5)
    to[short] (0,0)
    (0,0.3) to[open,v^>=$\cn{U}_{Z_1}(\cn{s})$] (3,0.3)
    (3.3,0) to[open,v^<=$\cn{U}_{Z_2}(\cn{s})$] (3.3,-3)
    (-0.3,0) to[open,v_<=$\cn{U}_{J}(\cn{s})$] (-0.3,-3);
  \draw
    (1,-1.5)[blue,->,>=latex] arc (180:-90:0.5) node[below] {$M$};
  \begin{scope}[yshift=-7cm]
  \draw(0,0) 
      node[draw,circle,inner sep=0.5mm,blue] [above left = 0.3] {$1$}
    to[short,o-] (0.5,0)
    to[R,l_=$\cn Z_1$] (2.5,0)
    to[short,-*,i=$\cn I_{Z_1}'(\cn{s})$] (3,0)
      node[draw,circle,inner sep=0.5mm,blue] [above right = 0.3] {$2$}
    to[short] (3,-0.5)
    to[R,l_=$\cn Z_2$] (3,-2.5)
    to[short,i<=$\cn I_{Z_2}'(\cn{s})$] (3,-3.0)
    to[short,-*] (0,-3.0)
      node[draw,circle,inner sep=0.5mm,blue] [below left = 0.3] {$3$}
    (0.5,0) -- (0.5,1.25) to[I_<=$\cn J$] (2.5,1.25) -- (2.5,0)
    (3,-0.5) -- (4.25,-0.5) to[I_<=$\cn J$] (4.25,-2.5) -- (3,-2.5)
%     to[short,i=$\cn I_J$] (0,-2.5)
%     to[I_=$\cn J$] (0,-0.5)
%     to[short] (0,0)
    (0,1.55) to[open,v^>=$\cn{U}_{Z_1}(\cn{s})$] (3,1.55)
    (4.55,0) to[open,v^<=$\cn{U}_{Z_2}(\cn{s})$] (4.55,-3);
%     (-0.3,0) to[open,v_>=$\cn{U}_{J}(\cn{s})$] (-0.3,-3)
  \draw
    (1,-1.5)[blue,->,>=latex] arc (180:-90:0.5) node[below] {$M$};
  \end{scope}
\end{circuitikz}
\end{minipage}
\begin{minipage}{0.5\textwidth}
\begin{enumerate}[label=\arabic*.)]
  \item
    Masche zeichnen, alle Zweigvariablen hinschreiben, ggf. selbst definieren
  \item
    Maschenrichtung so wählen, dass sie in Richtung der STQ zeigt (Hilfsmittel)
  \item
    STQ verschieben:
    \begin{itemize}
      \item "`Alte STQ"' $\to$ Leerlauf
      \item "`Neue STQ"' in alle anderen Zweige der Masche so einzeichnen,
        dass sie gegen Maschensinn gerichtet sind
      \item Neue Zweigströme einführen.
        Die Zweigspannungen ändern sich \emph{nicht}!
    \end{itemize}
\end{enumerate}
\end{minipage}

\paragraph{Trafo-Gl.:}
\[\begin{pmatrix}
    \cn{I}_{Z_1}' \\
    \cn{I}_{Z_2}'
  \end{pmatrix}(\cn{s})
  = \begin{pmatrix}
      \cn{I}_{Z_1} \\
      \cn{I}_{Z_2}
    \end{pmatrix}(\cn{s})
    +\cn{J}(\cn{s})
    \begin{pmatrix}-1\\1\end{pmatrix}
\]

\paragraph{Rück-Trafo:}
\[\begin{pmatrix}
    \cn{I}_{Z_1} \\
    \cn{I}_{Z_2}
  \end{pmatrix}(\cn{s})
  = \begin{pmatrix}
      \cn{I}_{Z_1}' \\
      \cn{I}_{Z_2}'
    \end{pmatrix}(\cn{s})
    -\cn{J}(\cn{s})
    \begin{pmatrix}-1\\1\end{pmatrix}\]

\paragraph{Rückgewinnungs-Gl.:}
\[\begin{aligned}
  \cn{I}_J(\cn{s})  &= \cn{J}(\cn{s}) \\
  \cn{U}_J(\cn{s})  &= -\cn{U}_{Z_1}(\cn{s}) + \cn{U}_{Z_2}(\cn{s})
\end{aligned}\]

% 
\chapter{Aufgabe zur komplexen PBZ}

\begin{circuitikz}
  
\end{circuitikz}

\begin{minipage}{0.5\textwidth}
Schaltplan
\end{minipage}
\begin{minipage}{0.5\textwidth}
\begin{itemize}
  \item $S$ ist ein idealer Umschalter, der bei $t=t_1$
    von der unteren in die obere Stellung wechselt
    und bei $t=t_2$ wieder zurückspringt
  \item $v(t)$ ist gegeben durch\\
    \begin{tikzpicture}[>=latex,scale=0.5]
      \draw[->] (-4,0) -- (7,0) node[below right] {$t$};
      \draw[->] (0,-1) -- (0,4.5) node[right] {$v(t)$};
      %
      \draw[thick] (-4,1)
        -- (-3,0)
        -- (1,0)
        -- (3,3)
        -- (3,1)
        -- (6,3)
        -- (6,0)
        -- (7,0);
      %
      \draw (-3,-0.2) -- + (0,0.4) node[below=1mm] {$t_1$};
      \draw (0,-0.2) -- + (0,0.4) node[below=1mm] {$t_2$};
      \draw (1,-0.2) -- + (0,0.4) node[below=1mm] {$T$};
      \draw (3,-0.2) -- + (0,0.4) node[below=1mm] {$3T$};
      \draw (6,-0.2) -- + (0,0.4) node[below=1mm] {$3T$};
      \draw (-0.2,1) -- + (0.4,0) node[left=1mm] {$\frac{V_0}{3}$};
      \draw (-0.2,3) -- + (0.4,0) node[left=1mm] {$V_0$};
    \end{tikzpicture}
  \item $R_1$, $C_1$, $L_1$, $L_2 > 0$ linear, zeitinvariant,
    $v(t)$ ideale, ungest. SPQ
\end{itemize}
\end{minipage}

Weitere Hinweise:
\begin{enumerate}
  \item $-t_1 >> RC+\frac{L_2}{R} > 0$
  \item Das NWM existiert für alle Zeiten und es gilt $e(t) = v(t)$,
    $a(t) = u_C(t)$ für alle Zeiten
  \item $\frac{L_1}{R} = \frac{43}{104}T$,
    $\frac{L_2}{R} = \frac{43}{111}T$,
    $CL_1 = \frac{1}{43}T^2$
  \item Für $t>0$ ist $\cn A_1 = -\frac{3}{T}<0$ eine nat. Freq.
\end{enumerate}

\textbf{Aufgabe:} Berechnen Sie $u_C(t)$ für \underline{alle} Zeiten.

Lösung zur Aufgabe mit komplexer PBZ

\begin{itemize}
  \item
    Die Anfangswerte sind weder für $t=t_{1,+,-}$ noch für
    $t=t_{2,+,-}$ gegeben\\
    $\Rightarrow$ Betrachte das NWM für $t<t_1$!
  \item
    Für $t<t_1$ und $t>t_2$ ist die Struktur der NWM identisch, dort gilt
    \begin{align*}
      \cn H(\cn s)
        &= \frac{\cn U(\cn s}{\cn V(\cn s)}
         = \frac{%
            R\parallel\left(\frac{1}{\cn sC}+\cn sL_1\right)%
          }{%
            R\parallel\left(\frac{1}{\cn sC}+\cn sL_2\right)
          }
          \cdot\frac{%
            \frac{1}{\cn sC}
          }{%
            \frac{1}{\cn sC}+\cn sL_1
          }\\
        &= -\frac{%
            R\cdot\cancel{\left(\frac{1}{\cn sC}+\cn sL_1\right)}
          }{%
            R\cdot\left(\frac{1}{\cn sC}+\cn sL_1\right)
            +\cn sL_2\left(\cn sL_1+R+\frac{1}{\cn sC}\right)
          }
          \cdot\frac{%
            \frac{1}{\cn sC}
          }{%
            \cancel{\frac{1}{\cn sC}+\cn sL_1}
          }\\
        &= -\frac{%
          R
          }{%
            R\left(\cn s^2CL_1+1\right)+\cn sL_2\left(\cn s^2L_1C+\cn sRC+1\right)
          }\\
        &= -\frac{%
          R
          }{%
            \cn s^3CL_1L_2+\cn s^2RC(L_1+L_2)+\cn sL_2+R
          }\\
        &= -\frac{R}{CL_1L_2}\cdot
          \frac{1}{%
            \cn s^3+\cn s^2\cdot\left(\frac{R}{L_1}+\frac{R}{L_2}\right)
            +\cn s\cdot \frac{1}{CL_1}+\frac{R}{CL_1L_2}
          }\\
        &= \frac{111\cdot \cancel{45}}{\cancel{45}\cdot T^2\cdot T}
          \cdot\frac{1}{%
            \cn s^3+\cn s^2\cdot\frac{\cancel{215}}{\cancel{43}T}+\cn s\cdot\frac{43}{T^2}
            +\frac{111\cdot 43}{43\cdot T^2\cdot T}
          }\\
        &\!\!\!\!\underset{x:=\cn sT}{=} 111\cdot\frac{1}{%
            x^3+5x^2+43x+111
          } = \frac{\cn P(x)}{\cn Q(x)}\\
    \end{align*}

\item $\cn P(s)$, $\cn Q(s)$ sind sicher teilerfremd; außerdem gilt für $t<t_1$
  $t>t_2$: $n_A=3$ ($u_c(t)$, $i_{L_1}(t)$, $i_{L_2}(t)$)\\
  $\Rightarrow$ $3=n_A\geq n\geq \grad(\cn Q(\cn s))=3\quad\Rightarrow\quad n=n_A=3$,
  $\cn Q(\cn s)$ liefert alle nat. Freq, das NWM besitzt keine inkonsitenten AW
  für $t=?$

\item Zeige: Alle nat. Freq haben einen neg. Realteil! Nach Aufgabe:
  $\cn s=-\frac{3}{T} \Rightarrow x_1 = -3$\\
  Mit Horner-Schema geht die Poly-Div schnell:\\
  \[\begin{tabular}{rrrr}
    $1$ & $5$ & $43$ & $111$\\
    \fbox{$-3$} & $-3$ & $-6$ & $-111$\\
    \midrule
    $1$ & $2$ & $37$ & \underline{\underline{$0$}}
  \end{tabular}\]

  $\Rightarrow$ $\cn Q'(x) = (x+3)\cdot (x^2+2x+37) = (x+3)\cdot ((x+1)^2 + 6^2)$
  mit den NSTen von $\cn Q(s)$:
  $\cn s_1 = -\frac{3}{T}$, $s_{2,3}=-\frac{2}{T}\pm j\frac{6}{T}$\\
  $\Rightarrow$ Alle nat. Freq. haben für $t<t_1$, $t>t_2$ einen neg. Realteil,
  das NWM ist dor asympt. stabil!

\item Für $t<t_1$ ist $v(t)=V_0/3$ eine DC-Quelle; das asymptot. stabile NWM
  ex. für alle Zeiten und befindet sich deshalb für $t<t_1$ im EZ;
  sogar im DC-EZ, weil für $t<t_1$ $v(t)$ eine DC-Quelle ist!

  Berechne $u_v(t)$ für $t<t_1$ per DC-ESB, ebenso alle AW für $t=t_1^-$:
  $u_c(t) = -\frac{V_0}{3}=u_c(t_1^-)$, $i_{L_1} = 0$, $i_{L_2} = -\frac{-V_0}{3R}$

\item Beweise Stabilität für $t_1 < t < t_2$; dort ist $v(t)=0$, führe die
  Zeitverschiebung $\tilde t:= t-t_1$ ein!
  \begin{align*}
    \cn H_1(\cn s) &:= \frac{\tilde{\cn U}_{C_1}(\cn s)}{C\cdot u_v(t_1^-)}
    = \frac{1}{\cn sC}\parallel R
    = \frac{R}{\cn sRC+1} = \frac{\cn P_1(\cn s)}{\cn Q_1(\cn s)} \text{ teilerfremd, }\\
    &\cn Q_1(\cn s)\stackrel{!}{=}0
    \Rightarrow \cn s=-\frac{1}{RC}<0\\
    \cn H_2(\cn s) &:= \frac{\tilde{\cn I}_{L_2}(\cn s)}{L_2\cdot i_{L_2}(t_1^-)}
    = \frac{1}{\cn sL_2+R}
    = \frac{\cn P_2(\cn s)}{\cn Q_2(\cn s)} \text{ teilerfremd, }\\
    &\cn Q_2(\cn s)\stackrel{!}{=}0
    \Rightarrow \cn s=-\frac{R}{L_2}<0
  \end{align*}
  Wegen $\frac{R}{L_2}=\frac{111}{43T} \neq \frac{43^2}{104T} = \frac{1}{RC}$
  sind $\cn Q_1(\cn s)$, $\cn Q_2(\cn s)$ teilerfremd.
  Es ist immer noch $n_A=3$, aber $n_R=1$ ($i_{L_1}(t) = 0$)\\
  $\Rightarrow$ $2=n_A-n_R\geq n \geq \grad(\cn Q_1(\cn s))+\grad(\cn Q_2(\cn s))=2$;
  $\cn Q_1(\cn s)$, $\cn Q_2(\cn s)$ liefern alle nat. Freq.
  Sie haben alle einen negativen Reatlteil (s.o.)
  $\Rightarrow$ Das NWM ist für $t_1<t<t_2$ asymptotisch stabil!
\end{itemize}


% Anhang
\appendix

\chapter{Kleine Formelsammlung}

\section{Tabelle wichtiger Laplace-Transformationen}\skilllabel{laplacetrafos}
\begin{center}
\begin{tabular}{ll}
$f(t)$  & $\mathfrak{L}\{f(t)\}$\\
\midrule
$\delta(t)$ & $1$\\[1ex]
$\delta^{(n)}(t)$ & $\cn s^n$\\[1ex]
$\dfrac{1}{\cn s+a}$ & $\Theta(t)\cdot e^{-at}$\\[1ex]
$\dfrac{a}{\cn s^2+a^2}$ & $\sin(at)$\\[1ex]
$\dfrac{\cn s}{\cn s^2+a^2}$ & $\cos(at)$\\[1ex]
\end{tabular}
\end{center}

\section{Additionstheoreme}\skilllabel{additionstheoreme}

\begin{align*}
  \sin(x)\cdot\cos(y) &= \frac{1}{2}(\sin(x-y)+\sin(x+y))\\
  \sin(x)\cdot\sin(y) &= \frac{1}{2}(\cos(x-y)-\cos(x+y))\\
  \cos(x)\cdot\cos(y) &= \frac{1}{2}(\cos(x-y)+\cos(x+y))\\
\end{align*}

\begin{align*}
  \cos^2(x) &= \frac{1}{2}(1+\cos(2x))\\
  \sin^2(x) &= \frac{1}{2}(1-\cos(2x))
\end{align*}


\chapter{Verzeichnisse}
%Glossar ausgeben
\printglossary[style=altlist,title=Glossar]
%Abkürzungen ausgeben
\printglossary[type=\acronymtype,style=long]
%Symbole ausgeben
\printglossary[type=symbolslist,style=long]
%Index
\printindex
\end{document}
