% \chapter{28.01.2011}

\chapter{Vierpole/Zweitore}
\Index{Vierpol}\Index{Zweitor}

\begin{minipage}{0.6\textwidth}\centering
\begin{circuitikz}
  \draw (0,0)
    -- (2,0)
    -- (2,3)
    -- (0,3)
    -- (0,0)
    (0,0.5) to[short,-o,i_>=$\cn{I'}_1$] (-1,0.5) node [below] {Tor 1}
    (0,2.5) to[short,-o,i_<=$\cn{I}_1$] (-1,2.5)
    (2,0.5) to[short,-o,i^>=$\cn{I'}_2$] (3,0.5) node [below] {Tor 2}
    (2,2.5) to[short,-o,i^<=$\cn{I}_2$] (3,2.5)
    (-1,0.5) to[open,v^=$\cn{U}_1$] (-1,2.5)
    (3,0.5) to[open,v_=$\cn{U}_2$] (3,2.5)
;\end{circuitikz}
\end{minipage}
\begin{minipage}{0.4\textwidth}
Definition eines Zweitors\\
(Richtung $\cn{I}_{1,2}$, $\cn{U}_{1,2}$!)
\end{minipage}\medskip

\paragraph{Matrixdarstellung:}
\[\begin{array}{clccc}
    \begin{pmatrix}\cn{A}_1\\\cn{A}_2\end{pmatrix}
    &= \begin{pmatrix}
        \cn{M}_{11} & \cn{M}_{12} \\
        \cn{M}_{21} & \cn{M}_{22}
      \end{pmatrix}
    &\cdot\begin{pmatrix}
          \cn{E}_1  \\  \cn{E}_2
         \end{pmatrix}
    &+&\begin{pmatrix}
        \cn{Q}_1  \\ \cn{Q}_2
      \end{pmatrix} \\
  \uparrow  && \uparrow &&\uparrow\\
  \begin{minipage}{3.5cm}\small\raggedright
    \text{Ausgangsvektor,}
    \begin{description}
      \item[$\cn{A}_1, \cn{A}_2\colon$]
      Ausgangs\-zeiger aus
      \{$\cn{I}_1$, $\cn{I}_2$, $\cn{U}_1$, $\cn{U}_2$\}
    \end{description}
  \end{minipage}
  &&\begin{minipage}{3cm}\small\raggedright
      Eingangsvektor,
      $\cn{E}_1$, $\cn{E}_2$ sind die beiden
      anderen Zeiger aus
      \{$\cn{I}_1$, $\cn{I}_2$, $\cn{U}_1$, $\cn{U}_2$\}
   \end{minipage}
   &&\begin{minipage}{1.5cm}\small\raggedright
      Vektor der festen Quellen
    \end{minipage}  \\
  \downarrow  &&  \downarrow \\
  \text{\small Rechnet man aus}
  &&\text{\small Gibt man vor}
  \end{array}\]

\paragraph{Vorgehensweise:}
\begin{enumerate}[label=\arabic*)]
  \item {\boldmath$\cn{Q}_1$, $\cn{Q}_2$} bestimmen:
    Setze $\cn{E}_1 = \cn{E}_2 \stackrel{!}{=} 0
    \Rightarrow \begin{pmatrix}\cn{A}_1\\\cn{A}_2\end{pmatrix}
    =\begin{pmatrix}\cn{Q}_1\\\cn{Q}_2\end{pmatrix}$
    \begin{example}
      \begin{tabular}[t]{ll}
        $\cn{E}_1$ ist eine Spannung: & KS an Tor 1!  \\
        $\cn{E}_2$ ist ein Strom:     & LL an Tor 2!  \\
      \end{tabular}
    \end{example}

  \item {\boldmath$\cn{M}_{11}$, $\cn{M}_{21}$} bestimmen:
    Alle festen Quellen $\stackrel{!}{=} 0 \Rightarrow \cn{Q}_1, \cn{Q}_2 = 0$
    
    $\cn{E}_2 \stackrel{!}{=} 0$
    \[\Rightarrow\begin{pmatrix}\cn{A}_1\\\cn{A}_2\end{pmatrix}
      = \begin{pmatrix}
          \cn{M}_{11}\cdot\cn{E}_1\\
          \cn{M}_{21}\cdot\cn{E}_1
        \end{pmatrix}
      \Rightarrow\begin{aligned}
        \cn{M}_{11} = \left.\frac{\cn{A}_1}{\cn{E}_1}\right|_{
          \cn{Q}_1=\cn{Q}_2=\cn{E}_2\stackrel{!}{=}0}\\
        \cn{M}_{21} = \left.\frac{\cn{A}_2}{\cn{E}_1}\right|_{
          \cn{Q}_1=\cn{Q}_2=\cn{E}_2\stackrel{!}{=}0}
      \end{aligned}\]

%     $\rightarrow \cn{E}_1\colon$
%     \begin{tabular}{@{}ll@{}}Stromquelle\\Spannungsquelle\end{tabular}
%     einzeichnen
% 
%     $\cn{E}_2$, alle festen
%     \begin{tabular}{@{}l@{}}Spannungsquellen\\Stromquellen\end{tabular}
%     werden zu
%     \begin{tabular}{@{}l@{}}KS\\LL\end{tabular}
    
    \begin{tabular}{l@{\,\,:\quad}l}
      $\cn{E}_1$ & \begin{tabular}{@{}ll@{}}Stromquelle\\Spannungsquelle\end{tabular}
      einzeichnen\\
      $\cn{E}_2$, alle festen
    \begin{tabular}{@{}l@{}}SPQ\\STQ\end{tabular} &
      werden zu
    \begin{tabular}{@{}l@{}}KS\\LL\end{tabular}
    \end{tabular}


  \item {\boldmath$\cn{M}_{12}$, $\cn{M}_{22}$} bestimmen:
    Alle festen Quellen $\stackrel{!}{=} 0 \Rightarrow \cn{Q}_1$, $\cn{Q}_2=0$
    
    $\cn{E}_1 \stackrel{!}{=} 0$
    \[\Rightarrow\begin{pmatrix}\cn{A}_1\\\cn{A}_2\end{pmatrix}
      = \begin{pmatrix}
          \cn{M}_{12}\cdot\cn{E}_2\\
          \cn{M}_{22}\cdot\cn{E}_2
        \end{pmatrix}
      \Rightarrow\begin{aligned}
          \cn{M}_{12} = \left.\frac{\cn{A}_1}{\cn{E}_2}\right|_{
            \cn{Q}_1=\cn{Q}_2=\cn{E}_1\stackrel{!}{=}0} \\
          \cn{M}_{22} = \left.\frac{\cn{A}_2}{\cn{E}_2}\right|_{
            \cn{Q}_1=\cn{Q}_2=\cn{E}_1\stackrel{!}{=}0}
        \end{aligned}
    \]
% 
%     $\cn{E}_1$, alle festen Quellen werden zu
%       \begin{tabular}{l}
%         KS\\LL
%       \end{tabular}
% 
%     $\rightarrow \cn{E}_2\colon$
%       \begin{tabular}{@{}ll@{}}Stromquelle\\Spannungsquelle\end{tabular} einzeichnen
      
    \begin{tabular}{l@{\,\,:\quad}l}
      $\cn{E}_1$, alle festen
      \begin{tabular}{@{}l@{}}SPQ\\STQ\end{tabular} &
      werden zu
      \begin{tabular}{@{}l@{}}KS\\LL\end{tabular}\\
      $\cn{E}_2$ & \begin{tabular}{@{}ll@{}}Stromquelle\\Spannungsquelle\end{tabular}
      einzeichnen
    \end{tabular}
\end{enumerate}

