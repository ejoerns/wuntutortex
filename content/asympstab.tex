% \chapter{12.11.2010}

\chapter{Asymptotische Stabilität}

\infobox*{\raggedright%
  Ein Netzwerkmodell ist asymptotisch stabil, wenn
  alle natürlichen Frequenzen des Netzwerkmodells einen
  \emph{negativen Realteil} haben.\\
  \raggedleft\small($\to$ Satz 5.2)}
Die nat. Freq. $\cn A_i$ bestimmen, ob, und wenn ja,
wie schnell das \acs{NWM} einschwingen \emph{kann}.

% \begin{itemize}
%     $\to$ Bsp.: Die Lösung des \acs{NWM}s liefert $e^{\cn{A}_i\cdot t}$,
%     $\cn{A}_i$: natürliche Frequenz, $\cn{A}_i = \alpha + j\beta$
%     \begin{description}
%       \item[NR:] $e^{\cn{A}_i t} = e^{\alpha t}\cdot e^{j\beta t}$;
%         $e^{\alpha t}$ klingt ab, wenn $\alpha<0$
%     \end{description}
% \end{itemize}


\section{Wie beweist man asymptotische Stabilität?}% margin: siehe FS

Zuerst müssen \emph{alle} ($n$) natürlichen Frequenzen gefunden werden (Punkt 1-3).
Dann wird bewiesen, dass sie einen negativen Realteil haben (Hurwitz-Kriterium)
(Punkt 4).



\begin{enumerate}
  \item $n_A$ und $n_R$ bestimmen!\hfill
  \begin{description}
    \item[$n_A:$] Anzahl der differenzierbaren Variablen,
      also Variablen, die abgeleitet werden.
      \begin{description}
        \item[Bsp.:]\hfill\\
          \begin{minipage}{0.5\textwidth}
            \begin{circuitikz}[scale=0.7,baseline=(current bounding box.center)]
              \ctikzset{voltage/distance from node=.3}
              \draw (0,0)
                to[C=$C$,i=$i_C(t)$,v>=$u_C(t)$] (0,-3);
            \end{circuitikz}
            $i_c(t) = C\cdot \frac{\partial}{\partial t}\colorbox{emphlight}{$u_C(t)$}$
          \end{minipage}
          \begin{minipage}{0.5\textwidth}
            \begin{circuitikz}[scale=0.7,baseline=(current bounding box.center)]
              \ctikzset{voltage/distance from node=.4}
              \draw (0,0)
                to[L=$L$,i=$i_L(t)$,v>=$u_L(t)$] (0,-3);
            \end{circuitikz}
            $u_L(t) = L\cdot \frac{\partial}{\partial t}\colorbox{emphlight}{$i_L(t)$}$
          \end{minipage}
          \par\vspace*{5mm}
          \begin{minipage}{0.5\textwidth}
            ~\,%
            \begin{circuitikz}[scale=0.7,baseline=(current bounding box.center)]
              \ctikzset{voltage/distance from node=.4}
              \draw (0,0)
                to[R=$R$,v>=$u_R(t)$,i={\,}] (0,-3);
              \draw (2,0)
                to[I] (2,-3);
            \end{circuitikz}
            $\alpha \frac{\partial}{\partial t} \colorbox{emphlight}{$u_R(t)$}$
          \end{minipage}
          \begin{minipage}{0.5\textwidth}
            \quad%
            \begin{circuitikz}[scale=0.7,baseline=(current bounding box.center)]
              \ctikzset{voltage/distance from node=.4}
              \draw (0,0)
                to[L=$L$,i>=$i_R(t)$] (0,-3);
              \draw (2,0)
                to[V] (2,-3);
            \end{circuitikz}
            $\beta\frac{\partial}{\partial t} \colorbox{emphlight}{$i_L(t)$}$
          \end{minipage}
      \end{description}
%       \begin{itemize}
%         \item Erst $u_C(t)$, $i_L(t)$ zählen
%         \item Dann die durch Ableitung gesteuerten Quellen.\\
%           Diese liefern nur
%           dann eine differenzierbare Variable, falls sie noch nicht gezählt wurden.\\
%           $\to$ \emph{Keine differenzierbare Variable wird "`doppelt"' gezählt}
%       \end{itemize}

    \item[$n_R:$] Anzahl der zustandsreduzierenden Gleichungen (ZRG)
      \begin{description}
        \item[Allgemein:]
          \[\sum_{i=1}^n \alpha_i\cdot x_i(t)
            + \sum_{i=1}^{n_Q} \beta_i\cdot e_i^{(k)}(t) = 0\]
          \begin{tabular}{r@{\,:~}l}
            $x_i(t)$      &differenzierbare Variable  \\
            $e_i^{(k)}$   &\emph{ungesteuerte} Quellen + deren Ableitungen \\
            $\alpha_i, \beta_i$ &Konstanten, z.B. $RCL$, etc.
          \end{tabular}
        \item[Spezialfälle:]\hfill
          \begin{itemize}
            \item MGL \emph{nur} aus $C$s und/oder festen SPQ
            \item SGL \emph{nur} aus $L$s und/oder festen STQ
          \end{itemize}
          \leftpointright~Gibt es \emph{keine} gesteuerten Quellen im NWM,
          kann es nur solche ZRG geben.
      \end{description}
      Wichtig: Für die Anzahl $n_{R_g}$ der \emph{gefundenen} ZRG gilt:
      \[n_R \geq n_{R_g}\qquad !\]
    \item[$n:$]\hfill
      \begin{itemize}
        \item Ordnung des Netzwerkmodells, $n = n_A-n_R$
        \item Anzahl an nat. Freq. des NWMs
          (die man auf neg. Realteil testen soll)
        \item Ordnung des DGL-Systems, also ZRM (Zustandsraummodell)
      \end{itemize}
    \item[$\Rightarrow$] \colorbox{emphlight}{$n = n_A-n_R$}
  \end{description}

  \item ÜF/FG aufstellen:
    \[\cn{H}(\cn{s}) = \frac{\cn{P}(\cn{s})}{\cn{Q}(\cn{s})}\ ;\quad
      \cn{P}(\cn{s}), \cn{Q}(\cn{s})\text{ Polynome}
    \]
    Sind $\cn{P}(\cn{s})$, $\cn{Q}(\cn{s})$ teilerfremd?\\
    \begin{tabular}{@{$\rightarrow$\,}r@{\,:~\,}l@{}}
      Ja    & weiter, alle Nullstellen von $\cn{Q}(\cn{s})$ sind
        natürliche Frequenzen\\
      Nein  & kürzen
    \end{tabular}
    \begin{description}
      \item[Anmerkung:] Hat man mehrere ÜF, müssen auch die $\cn{Q}_i(\cn{s})$
        zueinander teilerfremd sein
    \end{description}

  \item Es gilt $n_A-n_R = n \geq \grad(\cn{Q}(\cn{s}))$.
    Gilt auch $n_A-n_R=\grad(\cn{Q}(\cn{s}))$?\\
    \begin{tabular}{@{$\rightarrow$\,}r@{\,:~\,}l@{}}
      Ja    & weiter  \\
      Nein  & Neue ÜF aufstellen, die (hoffentlich)
        alle fehlenden nat. Freq. (zu 2.) liefert
    \end{tabular}

  \item Haben alle Nullstellen von $\cn{Q}(\cn{s})$ (bzw. $\cn{Q_i}(\cn{s})$)
    einen negativen Realteil?\\
    \begin{tabular}{@{$\rightarrow$\,}r@{\,:~\,}l@{}}
      Ja    & Das NWM ist asymptotisch stabil  \\
      Nein  & Das NWM ist instabil
    \end{tabular}
    \begin{description}
      \item[Anmerkung:]
        $\cn{Q}(\cn{s})$ heißt \emph{Hurwitzpolynom}, falls alle Nullstellen
        einen negativen Realteil haben.
        \Index{Hurwitzpolynom}
        
        Für $\grad(\cn{Q}(\cn s)) \leq 2$ gilt:
        \begin{empheq}[box=\colorbox{emphlight}]{align*}
          \text{Alle Koeffizienten sind }\begin{matrix}\text{positiv}\\[-0.5ex]\text{negativ}\end{matrix}
            &\Leftrightarrow \cn{Q}(\cn{s})\text{ ist ein Hurwitzpolynom }\\
            &\Leftrightarrow\text{Alle Nullstellen haben negativen Realteil}
          \end{empheq}
    \end{description}

\end{enumerate}

% TODO:
% \subsection{Teilerfremdheit}
% Einfacher ist es, Zähler-Nullstellen in den Nenner einzusetzen.
% Beim mehreren Q(s) testen, ob Q(s)s untereinander teilerfremd.


% \begin{description}
%   \item[Anmerkung:] Bei mehreren \acs{ZRG} muss man prüfen,
%     ob sie \emph{linear unabhängig} sind\\
%     $\to$ ob eine davon durch die anderen darstellbar ist
% \end{description}

\clearpage
\section{Beispiel: F08 A1}

\begin{figure}[h]
\begin{center}
\begin{circuitikz}[scale=1.2]\draw
	(0,0) -- (6,0)
	to [C,l_=$C_2$,v^>=$u_{C_2}(t)$] (6,-4)
	-- (4,-4)
	to [L=$L$, *-] (4,-2)
	to [R,l_=$R_2$,v^>=$u_{R_2}(t)$, -*] (4, 0)
	(4,-4)
	-- (2,-4)
	to[R=$R_1$,*-] (2,-2)
	-- (0,-2)
	to [C=$C_1$,v=$u_{C_1}(t)$,*-] (0,-4)
	-- (2,-4)
	(0,-2)
	to [V<=$v(t)$] (0,0);
\draw (2,-4) to [very thick] (4,-4);
\end{circuitikz}
\end{center}
\caption*{$R_i = R$, $C_i = C$}
\end{figure}

\paragraph{Aufgabe}
\begin{enumerate}[label=\alph{*})]
	\item $\cn{H}(j\omega) = \frac{\cn{P}(j\omega)}{\cn{Q}(j\omega)} =
\frac{\cn{U}_{R_2}}{\cn{V}}$
	\item Ist das \acs{NWM} asymptotisch stabil?
\end{enumerate}

% \infobox{Beispiel für zustandsreduzierende Gleichungen}{%
% 
% \begin{compactitem}
% 	\item Maschen aus $C$ und/oder festen SPQ
% 	\item Schnittmengen aus L und/oder festen STQ
% \end{compactitem}
% 
% \paragraph{Allgemein:}
% 
% $x_1(t), \hdots, x_n(t)$ sind diffbare Variablen,
% $e_1(t), \hdots, e_{n_Q}(t)$ sind feste Quellen
% \[
% \Rightarrow\quad \sum_{i=1}^n \alpha_i \cdot x_i(t) + \sum_{i=1}^{n_Q}\sum_{k=0}^{r_i} e_i^{(k)}(t) \cdot \beta_{i,k} = 0
% \]
% ist eine allgemeine zustandsreduzierende Gleichung.
% }

\paragraph{Lösung}
\begin{enumerate}[label=\arabic{*})]

  \item $n_A$, $n_R$ aufstellen:\par
  $n_A\colon n_A=3$, denn $u_{C_1}(t), u_{C_2}(t), i_L(t)$
    sind diffbare Variable!\\
  $n_R\colon n_R\geq 1$, wegen Masche aus $C_1, C_2, v(t)$% TODO: n_R\geq 1?

  \acs{MGL}: $u_{C_1}(t) + u_{C_2}(t) - v(t) = 0$
  \begin{description}
    \item[Beispiel:] AW von $u_{C_1}(t)$ für $t = t_0$ vorgegeben\\
      $\Rightarrow u_{C_2}(t_0) = v(t_0) - u_{C_1}(t_0)$
      ist \emph{nicht} frei vorgebbar! (1 Zustand fällt weg!)
  \end{description}
  $\Rightarrow n = n_A- n_R \leq 2$

  \item $\cn H(\cn s)$ aufstellen. Zeigen, dass $\cn{P}(\cn s)$, $\cn{Q}(\cn s)$
    teilerfremd sind:
    %
    \[\text{Aus a) berechnet:}\quad
      \cn H(\cn s) = \frac{\cn sCR^2+R}{\cn s^2 2RCL+\cn s\left(2R^2C+L\right)+2R}\]
    Zählerpolynom zu Null setzen und Ergebnis in Nennerpolynom einsetzen:
    \begin{align*}
      \cn{P}(\cn s) &\stackrel{!}{=} 0 \Rightarrow \cn s_0 = -\frac{1}{RC}\\
      \quad\cn{Q}\left(-\frac{1}{RC}\right)% TODO: \cn s -> \cn s !?
      &= \left(-\frac{1}{RC}\right)^2\cdot 2RCL +
        \left(-\frac{1}{RC}\right)\cdot (2R^2C+L) + 2R\\
      &= \frac{2L}{RC} - \frac{L}{RC} - 2R+2R
        = \frac{L}{RC} \underset{R,C,L > 0}{>} 0
    \end{align*}
    $\Rightarrow$ $\cn{P}(\cn s)$, $\cn{Q}(\cn s)$ sind teilerfremd,
    $\cn{Q}(\cn s)$ liefert 2 natürliche Frequenzen!


  \item Gilt $n=\grad(\cn Q(\cn s))?$\par
    $\Rightarrow 2\geq n_A - n_R = n \geq \grad\left(\cn{Q}(\cn s)\right) = 2$

    $\Rightarrow n = \grad\left(\cn{Q}(\cn s)\right) = 2, \cn{Q}(\cn s)$
    liefert \emph{alle} natürlichen Frequenzen des NWM!

  \item Haben alle NS von $\cn Q(\cn s)$ einen negativen Realteil?\par
    $\cn{Q}(\cn s)$ ist ein Hurwitzpolynom 2. Grades,
    daher haben alle Nullstellen einen echt negativen Realteil,
\end{enumerate}
$\rightarrow$ Das NWM ist \textbf{asymptotisch stabil}. \qed
