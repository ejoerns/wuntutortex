% \chapter{11.02.2011}

\chapter{Quellenverschiebung}

\emph{Wann muss man Quellen verschieben?}

\begin{minipage}{0.5\textwidth}
\quad\begin{tabular}{@{}ll}
  Maschen-$\cn{Z}$-Verfahren:       & STQ \\
  Schnittmengen-$\cn{Y}$-Verfahren: & SPQ \\
  Knoten-$\cn{Y}$-Verfahren:        & SPQ
\end{tabular}
\end{minipage}
\begin{minipage}{0.5\textwidth}
  aber \emph{nur}, wenn sie allein in einem Zweig vorkommen
  (d.h. \emph{ideale} Quellen)! Sonst: Quellensubstitution
\end{minipage}

\emph{Wie verschiebt man?}

\begin{minipage}{0.5\textwidth}
\quad\begin{tabular}{@{}l}SPQ\\STQ\end{tabular}verschiebt
man in \begin{tabular}{@{}l@{}}Schnittmengen\\Maschen\end{tabular}\,!
\end{minipage}
\begin{minipage}{0.5\textwidth}
\begin{tabular}{@{$\to$\,}l}
  \emph{Nur} die Zweigspannungen ändern sich!\\
  \emph{Nur} die Zweigströme ändern sich!\\
\end{tabular}
\end{minipage}


\emph{Welche Gleichungen muss man aufstellen?}

\begin{description}
  \item[Transforamationsgleichungen:]
    Neue Zweiggrößen (gestrichene Größen) durch die alten Zweiggrößen
    oder/und feste Quellen ausdrücken
    
    $\Rightarrow$ 1 Gl. pro Zweig, in den man eine Quelle hinein verschiebt.
  
  \item[Rücktransformationsgleichungen:]
    Alte Zweiggrößen durch neue Zweiggrößen und/oder feste Quellen ausdrücken,
    $\Rightarrow$ Trafo-Gleichungen umstellen!
    
    $\Rightarrow$ 1 Gl. pro Zweig, in den man eine Quelle hinein verschiebt.
    
  \item[Rückgewinnungsgleichungen:]
    Zweigspannung und -strom der Quelle berechnen, die man verschiebt.
    
    \begin{tabularx}{\textwidth}{lX}
      Bei SPQ: & Zweigspannung $\stackrel{!}{=}$ Quellspannung,\newline
        Zweigstrom durch SGL in die man verschiebt darstellen!\\
      Bei STQ: & Zweigstrom $\stackrel{!}{=}$ Quellstrom,\newline
        Zweigspannung durch MGL in die man verschiebt darstellen! 
    \end{tabularx}
    $\Rightarrow$ 2 Gleichungen
    
    \klausurhinweis{%
      Die ist notwendig, wenn in der Aufgabe steht:
        \glqq Geben Sie die vollständigen Quellentransformationsbeziehungen an\grqq.
      Sonst müssen nur die Gleichungen aufgeschrieben werden, die man später braucht
      (Steuergrößen/zu berechnende Größen).
    }
\end{description}
\paragraph{Bemerkungen}
\begin{itemize}
  \item
Steuerungen transformieren mit Rücktrafo-/Rückgewinnungs-Gl., falls nötig

Evtl.: Steuerungstransformation erst einsetzen, wenn man die Matrix schon
aufgestellt hat!

%   \item
% Man muss \begin{tabular}{l}SPQ\\STQ\end{tabular} beim
% \begin{tabular}{l}Schnittmengen-\\Maschenimpedanz-\end{tabular}Verfahren
% verschieben, wenn sie in einem Zweig allein ($\rightarrow$ ideal) vorkommen:
% %%%
% bzw.
% %%%
% , gesteuert oder ungesteuert

%   \item
% Alle Trafo-/Rücktrafo, Rückgewinnungsgl: "`Stellen Sie die vollst.
% Quellentrafos auf!"'
\end{itemize}


\section{SPQ-Verschiebung}

\begin{minipage}{0.5\textwidth}
\begin{circuitikz}
  \draw(0,0.5)
      node[draw,circle,inner sep=0.5mm,blue] [above left = 0.3] {$1$}
    to[short,o-,i_=$\cn{I}_{V}(\cn{s})$] (0,0)
    to[V,v=$\cn{V}(\cn{s})$] (0,-2)
    to[short,-*] (0,-2.5)
      node[draw,circle,inner sep=0.5mm,blue] [left = 0.7] {$2$}
    to[short] (0,-3)
    to[R=$R_2$] (0,-5)
    to[short,-o,i_=$\cn{I}_{R_2}(\cn{s})$] (0,-5.5)
      node[draw,circle,inner sep=0.5mm,blue] [below left = 0.3] {$3$}
    (0,-2.5)
    to[short] (0.5,-2.5)
    to[R,l_=$R_1$] (2.5,-2.5)
    to[short,-o,i_=$\cn{I}_{R_1}(\cn{s})$] (3.0,-2.5)
      node[draw,circle,inner sep=0.5mm,blue] [above right = 0.3] {$4$}
    (-0.3,0) to[open,v_>=$\cn{U}_{V}(\cn{s})$] (-0.3,-2.5)
    (-0.3,-2.5) to[open,v_>=$\cn{U}_{R_2}(\cn{s})$] (-0.3,-5.5)
    (0,-2.2) to[open,v^>=$\cn{U}_{R_1}(\cn{s})$] (3.0,-2.2)
    ;
  \draw
    (0,-2.5)[dashed,blue] circle (0.5);
  \draw
    ($(0,-2.5)+(50:0.5)$) [dashed,blue,<-,>=latex] -- +(50:0.5);

  \begin{scope}[yshift=-7.5cm]
  \draw(0,0) 
      node[draw,circle,inner sep=0.5mm,blue] [above = 0.7] {$2$}
      node[draw,circle,inner sep=0.5mm,blue] [left = 0.7] {$1$}
    to[short,*-] (0,-0.5)
    to[V_=$\cn{V}(\cn{s})$] (0,-2.5)
    to[R=$R_2$] (0,-4.5)
    to[short,-o,i_=$\cn{I}_{R_2}(\cn{s})$] (0,-5)
      node[draw,circle,inner sep=0.5mm,blue] [below left = 0.3] {$3$}
    (0,0)
    to[short] (0.5,0)
    to[V_=$\cn{V}(\cn{s})$] (2.5,0)
    to[R=$R_1$] (4.5,0)
    to[short,-o,i_=$\cn{I}_{R_1}(\cn{s})$] (5,0)
      node[draw,circle,inner sep=0.5mm,blue] [above right = 0.3] {$4$}
    (0,0.7)
    to[open,v^>=$\cn{U}_{R_1}'(\cn{s})$] (5,0.7)
    (0.7,0)
    to[open,v^>=$\cn{U}_{R_2}'(\cn{s})$] (0.7,-5);
  \draw
    (0,0)[dashed,blue] circle (0.5);
  \draw
    ($(0,0)+(50:0.5)$) [dashed,blue,<-,>=latex] -- +(50:0.5);
  \end{scope}
\end{circuitikz}
\end{minipage}
\begin{minipage}{0.5\textwidth}
\begin{enumerate}[label=\arabic*.)]
  \item
    Schnittmenge neu hinzeichnen, alle Zweiggrößen einzeichnen,
    ggf. selbst die Richtung definieren
  \item
    Schnittmengenrichtung so wählen, dass sie in Richtung der SPQ zeigt (Hilfsmittel)
  \item
    SPQ verschieben:
    \begin{itemize}
      \item Alte SPQ $\to$ Kurzschluss\\
        $\to$ Betreffende Knoten fallen zusammen
      \item Neue SPQ in alle anderen Zweige der Schnittmenge so einzeichnen, dass
        sie der Schnittmengenrichtung \emph{entgegen} gerichtet sind
      \item Neue Zweigspannungen einführen.
        Die Zweigströme ändern sich \emph{nicht}!
    \end{itemize}
\end{enumerate}
\end{minipage}

\paragraph{Trafo-Gl.:}
\[\begin{pmatrix}
    \cn{U}_{R_1}' \\
    \cn{U}_{R_2}'
  \end{pmatrix}(\cn{s})
  = \begin{pmatrix}
      \cn{U}_{R_1} \\
      \cn{U}_{R_2}
    \end{pmatrix}(\cn{s})
    +\cn{V}(\cn{s})
    \begin{pmatrix}1\\1\end{pmatrix}
\]

\paragraph{Rück-Trafo:}
\[\begin{pmatrix}
    \cn{U}_{R_1} \\
    \cn{U}_{R_2}
  \end{pmatrix}(\cn{s})
  = \begin{pmatrix}
      \cn{U}_{R_1}' \\
      \cn{U}_{R_2}'
    \end{pmatrix}(\cn{s})
    -\cn{V}(\cn{s})
    \begin{pmatrix}1\\1\end{pmatrix}\]

\paragraph{Rückgewinnungs-Gl.:}
\[\begin{aligned}
  \cn{U}_V(\cn{s})  &= \cn{V}(\cn{s}) \\
  \cn{I}_V(\cn{s})  &= \cn{I}_{R_1}(\cn{s}) + \cn{I}_{R_2}(\cn{s})
\end{aligned}\]


\section{STQ-Verschiebung}

\begin{minipage}{0.5\textwidth}
\begin{circuitikz}
  \draw(0,0) 
      node[draw,circle,inner sep=0.5mm,blue] [above left = 0.3] {$1$}
    to[short,o-] (0.5,0)
    to[R,l_=$\cn Z_1$] (2.5,0)
    to[short,-*,i=$\cn I_{Z_1}(\cn{s})$] (3,0)
      node[draw,circle,inner sep=0.5mm,blue] [above right = 0.3] {$2$}
    to[short] (3,-0.5)
    to[R,l_=$\cn Z_2$] (3,-2.5)
    to[short,i<=$\cn I_{Z_2}(\cn{s})$] (3,-3.0)
    to[short,-*] (0,-3.0)
      node[draw,circle,inner sep=0.5mm,blue] [below left = 0.3] {$3$}
    to[short,i=$\cn I_J(\cn{s})$] (0,-2.5)
    to[I_=$\cn J$] (0,-0.5)
    to[short] (0,0)
    (0,0.3) to[open,v^>=$\cn{U}_{Z_1}(\cn{s})$] (3,0.3)
    (3.3,0) to[open,v^<=$\cn{U}_{Z_2}(\cn{s})$] (3.3,-3)
    (-0.3,0) to[open,v_<=$\cn{U}_{J}(\cn{s})$] (-0.3,-3);
  \draw
    (1,-1.5)[blue,->,>=latex] arc (180:-90:0.5) node[below] {$M$};
  \begin{scope}[yshift=-7cm]
  \draw(0,0) 
      node[draw,circle,inner sep=0.5mm,blue] [above left = 0.3] {$1$}
    to[short,o-] (0.5,0)
    to[R,l_=$\cn Z_1$] (2.5,0)
    to[short,-*,i=$\cn I_{Z_1}'(\cn{s})$] (3,0)
      node[draw,circle,inner sep=0.5mm,blue] [above right = 0.3] {$2$}
    to[short] (3,-0.5)
    to[R,l_=$\cn Z_2$] (3,-2.5)
    to[short,i<=$\cn I_{Z_2}'(\cn{s})$] (3,-3.0)
    to[short,-*] (0,-3.0)
      node[draw,circle,inner sep=0.5mm,blue] [below left = 0.3] {$3$}
    (0.5,0) -- (0.5,1.25) to[I_<=$\cn J$] (2.5,1.25) -- (2.5,0)
    (3,-0.5) -- (4.25,-0.5) to[I_<=$\cn J$] (4.25,-2.5) -- (3,-2.5)
%     to[short,i=$\cn I_J$] (0,-2.5)
%     to[I_=$\cn J$] (0,-0.5)
%     to[short] (0,0)
    (0,1.55) to[open,v^>=$\cn{U}_{Z_1}(\cn{s})$] (3,1.55)
    (4.55,0) to[open,v^<=$\cn{U}_{Z_2}(\cn{s})$] (4.55,-3);
%     (-0.3,0) to[open,v_>=$\cn{U}_{J}(\cn{s})$] (-0.3,-3)
  \draw
    (1,-1.5)[blue,->,>=latex] arc (180:-90:0.5) node[below] {$M$};
  \end{scope}
\end{circuitikz}
\end{minipage}
\begin{minipage}{0.5\textwidth}
\begin{enumerate}[label=\arabic*.)]
  \item
    Masche zeichnen, alle Zweigvariablen hinschreiben, ggf. selbst definieren
  \item
    Maschenrichtung so wählen, dass sie in Richtung der STQ zeigt (Hilfsmittel)
  \item
    STQ verschieben:
    \begin{itemize}
      \item "`Alte STQ"' $\to$ Leerlauf
      \item "`Neue STQ"' in alle anderen Zweige der Masche so einzeichnen,
        dass sie gegen Maschensinn gerichtet sind
      \item Neue Zweigströme einführen.
        Die Zweigspannungen ändern sich \emph{nicht}!
    \end{itemize}
\end{enumerate}
\end{minipage}

\paragraph{Trafo-Gl.:}
\[\begin{pmatrix}
    \cn{I}_{Z_1}' \\
    \cn{I}_{Z_2}'
  \end{pmatrix}(\cn{s})
  = \begin{pmatrix}
      \cn{I}_{Z_1} \\
      \cn{I}_{Z_2}
    \end{pmatrix}(\cn{s})
    +\cn{J}(\cn{s})
    \begin{pmatrix}-1\\1\end{pmatrix}
\]

\paragraph{Rück-Trafo:}
\[\begin{pmatrix}
    \cn{I}_{Z_1} \\
    \cn{I}_{Z_2}
  \end{pmatrix}(\cn{s})
  = \begin{pmatrix}
      \cn{I}_{Z_1}' \\
      \cn{I}_{Z_2}'
    \end{pmatrix}(\cn{s})
    -\cn{J}(\cn{s})
    \begin{pmatrix}-1\\1\end{pmatrix}\]

\paragraph{Rückgewinnungs-Gl.:}
\[\begin{aligned}
  \cn{I}_J(\cn{s})  &= \cn{J}(\cn{s}) \\
  \cn{U}_J(\cn{s})  &= -\cn{U}_{Z_1}(\cn{s}) + \cn{U}_{Z_2}(\cn{s})
\end{aligned}\]
