% \chapter{14.01.2011}

\chapter{Transistorschaltungen}
\Index{Transistorschaltung}

\paragraph{Allgemeine Vorgehensweise}

\begin{enumerate}
	\item \acs{AP} ausrechnen (immer mit einem DC-\acs{ESB}!)
	\item Kleinsignalparameter aus dem Kennlinienfeld ablesen
	\item \acs{KSig}-\acs{ESB} zeichnen
\end{enumerate}
\Index{Arbeitspunkt}
\Index{Kleinsignalparameter}
\Index{Kleinsignal-Ersatzschaltbild}

\section{AP ausrechnen}
\Index{Arbeitspunkt}

Alle Zweiggrößen des \acs{NWM} sind Gleichgrößen
	\begin{itemize}
		\item $C \rightarrow$ Leerlauf
		\item $L \rightarrow$ Kurzschluss
		\item Gesteuerte
			\begin{tabular}{l}
				SPQ\\ STQ
			\end{tabular}, die durch Ableitungen gesteuert werden, werden durch
			\begin{tabular}{l}
			Kurzschluss\\ Leerlauf
			\end{tabular} ersetzt.

\begin{description}\item[Beispiel:]\hfill\\[-2ex]
\begin{center}
\begin{circuitikz}[scale=1.2]
\draw
	(0,0) to[V=$v(t)\eq\alpha\cdot\frac{\partial}{\partial t}u_R(t)$,*-*] (0,-2)
	
	(5,0) to[R,v^=$U_R(t)  \underset{\text{DC-Fall}}{\eq} U_R$,*-*] (5,-2)
	
	(0,-2.5) node[anchor=west] {$\widehat{=}\; v(t) = \alpha \cdot 0 = 0$}
	
	(0,-3) to[short,*-*] (0,-4);
\end{circuitikz}
\end{center}
\end{description}

\item Kleinsignalquellen $\left(v_{KS}(t), j_{KS}(t)\right)$ werden zu 0
gesetzt.
	\begin{figure}[h]
\begin{center}
\begin{circuitikz}[scale=1.2]
\draw
	(0,0) to[V=$v(t) \eq V_{DC} + v_{in,KS}(t)$,*-*] (0,-2)
	
	(6,0) to[V=$V_{DC}$,*-*] (6,-2)
	(4,-1) node[anchor=west] {$\underset{\text{DC-Fall}}{\longrightarrow}$};
\end{circuitikz}
\end{center}
\end{figure}
  \renewcommand{\labelitemi}{$\Rightarrow$}
  \item Nur DC-Quellen, gesteuerte Quellen ohne Ableitung,
  Widerstände und Transistoren im DC-ESB!
  (NWM im zeitunabhängigen eingeschwungenen Zustand)
\end{itemize}


% \newpage

	\begin{figure}[h]
\caption*{DC-ESB}
\begin{center}
\begin{circuitikz}[scale=0.8]
\draw
	(-1.5,1.5) node[pigfete,anchor=G] (tr) {}
	(tr.G) node[anchor=south] {G}
	(tr.D) node[anchor=east] {D}
	(tr.S) node[anchor=west] {S}
	
  % NW 2
  (-6,-1) |- (-4,1) |- (-6,-1)
  (-5,0) node {NW2}
  % NW 1
  (3, -1) |- (5, 1) |- (3,-1)
  (4,0) node {NW1}

  (-5,-1.5) -- (-5,-1)
  (-5,1) -- (-5,1.5)
    -- (-2,1.5) to[short,o-] (tr.G)
  (-0.25, -1.5)
    to[short, -o] (1.5,-1.5) -- (4, -1.5) -- (4, -1)
% 	(4,1) -- (4,2.75) -- (1.5, 2.75) to[short, o-,i_=$I_D$] (tr.D)
  (tr.D) to [short,-o,i_<=$I_D$] ($(tr.D)+(2,0)$) -| (4,1)
  (tr.S) to[short, -o] (-0.25, -1.5) -- (-5,-1.5)

	(1.5, 2.75) to[open,v^>=$U_{DS}$] (1.5, -1.5)
	(-2,1.5) to[open,v>=$U_{GS}$] (0, -1.5)
	;
\end{circuitikz}
\end{center}

\end{figure}


Bei MOS-Transistoren: $I_{G,DC} = 0$

\paragraph[AG]{\acs{AG} 1:}


Bei einfachen NWM: $I_D = f(U_{DS})$

\begin{itemize}
	\item Maschengleichung $M$ aufstellen und nach $I_D$ auflösen
	\item NW1 als \acs{ESPQ} darstellen
	\subitem $\to$ \acs{LL}-Spannung von NW1 bzgl: $D$ und $S$
\end{itemize}


\paragraph[AG]{\acs{AG} 2:}

\begin{itemize}
	\item Linke Seite genauso mit $I_G = 0$
	\item NW2 als \acs{ESPQ} bzgl. $G$ und $S$
\end{itemize}

% \begin{figure}
% \caption*{Grausames bild}
\begin{center}
\begin{tikzpicture}[scale=1.2]
  % Koordinatensystem
  \draw[->,>=latex] (0,0) node [below=0.4] {(NMOS)}
    -- (7,0) node [below right] {$U_{DS}$};
  \draw[->,>=latex] (0,0) -- (0,4) node [above left] {$I_{DS}$};
  \draw[->,>=latex] (7.5,3.0) -- node [right] {$U_{GS}$} (7.5,4.0);
  % Kennlinie
  \draw (0,0) to[out=60,in=190] (2.5,2.6) -- +(10:4);
  \draw (0,0) to[out=50,in=190] (2.5,2.0) -- +(10:4);
  \draw (0,0) to[out=40,in=190] (2.5,1.4) -- +(10:4) node [right] {$U_{GS}=1V$};
  \draw (0,0) to[out=30,in=190] (2.5,0.8) -- +(10:4);
  % AGs
  \draw[blue] (-0.1,3) node [left] {$\frac{V_{LL,NW1}}{R_{,NW1}}$}
    -- (0,3)
    -- (6,0)
    -- (6,-0.1) node [below] {$V_{LL,NW1}$};
  % Kreuze vertikal
  \node[cross out,draw,blue]  at ($(2.5,0.8)+(10:0.52)$) {};
  \node[cross out,draw,blue]  at ($(2.5,1.4)+(10:0.52)$) {};
  \node[cross out,draw,blue]  at ($(2.5,2.0)+(10:0.52)$) {};
  \node[cross out,draw,blue]  at ($(2.5,2.6)+(10:0.52)$) {};
  \draw[blue] ($(2.5,2.6)+(10:0.52)+(90:0.5)$) -- +(-90:3.5);
  % Tangente
  \draw[blue] ($(2.5,1.4)+(10:0.52)+(10:-3.2)$) node [left] {Tangente}
    -- +(10.1:6);
\end{tikzpicture}
\end{center}
% \end{figure}

\section{KS-Parameter}

\begin{align*}
  R_D^{-1}
    &= \left.\left(\frac{\partial I_D}{\partial U_D}\right)(AP)\right|_{
      U_G=\text{konst}=U_{G,DC}}
    &&\approx \frac{\triangle I_{DS}}{\triangle U_{DS}}
      \quad \text{der Tangente}\\
  g
    &= \left.\left(\frac{\partial I_D}{\partial U_G}\right)(AP)\right|_{
      U_D=\text{konst}=U_{D,DC}}
    &&\approx \frac{\triangle I_{DS}}{\triangle U_G}
      \quad \text{der Sekante} = \frac{I_{D,x1}-I_{D,x2}}{U_{G,x1}-U_{G,x2}}
\end{align*}

Aus AG2: $U_{GS}$ direkt ablesen (in den meisten Fällen) z.\,B. $U_{GS} =
1\text{V}$

Allgemein: $I_{DS}= f(U_{GS}) \Rightarrow$ Kennlinie Punktweise einzeichnen. Der Schnittpunkt beider Kennlinien ist der AP!

%%% HIER FEHLT NOCH EINIGES!

\paragraph{Näherung: Schaltzwang}\hfill

\begin{minipage}{0.5\textwidth}\centering
\begin{circuitikz}
  \draw node[nmos, anchor=D] (pmos) at (0,0) {}
    (pmos.D) node [right] {D}
    (pmos.G) node [below] {G}
    (pmos.S) node [right] {S}
    (pmos.D) to[short,i<=$I_D$] (0,0.75) to[short,*-o] (0,1.5)
    (pmos.S) to[short,-o] (0,-2)
    (0,0.75) -| ($(pmos.G)+(-0.6,0)$) to[short,i=$I_G$] (pmos.G)
;\end{circuitikz}
\end{minipage}
\begin{minipage}{0.5\textwidth}
\[U_{DS}\stackrel{!}{=}U_{GS}\]
Man kann sich nur auf dieser Kennlinie bewegen. $\rightarrow$ Einzeichnen!
\end{minipage}


%%% HIER FEHLT NOCH EINIGES!


\begin{circuitikz}
  \draw (0,0) node [circle,draw,inner sep=0.5mm,left=0.3] {S}
    to[R=$R_{ges}$,o-o] (3,0)
      node [circle,draw,inner sep=0.5mm,above right=0.3] {G}
      node [circle,draw,inner sep=0.5mm,below right=0.3] {D}
;\end{circuitikz}


\paragraph{"`Normalfall"'}\hfill

\begin{circuitikz}
  \draw (0,0)
      to[open,o-o] (0,2)
        node [draw,circle,inner sep=0.5mm,left=0.3] {G}
    (0,0) to[short,o-o] (5,0)
        node [circle,draw,inner sep=0.5mm,right=0.3] {S}
    (3,2) to[I_=$g\cdot U_{GS}(t)$,*-] (3,0)
    (4,2) to[R=$R_D$,*-*] (4,0)
    (3,2) to[short,-o] (5,2) node [circle,draw,inner sep=0.5mm,right=0.3] {D}
;\end{circuitikz}

\section{Kleinsignal-ESB}

\begin{enumerate}[label=\alph*)]
  \item KS-ESB der Schaltung\\
    $\rightarrow$ Transistor-KS-ESB einsetzen\\
    $\rightarrow$ DC-Quellen $\stackrel{!}{=}0$ setzen.
  \item fertig!!
\end{enumerate}

\paragraph{Anmerkung}

H04, F05, H05 sind Einsteigeraufgaben.

Teil d) kann erst beantwortet werden, wenn e) gelöst wurde


