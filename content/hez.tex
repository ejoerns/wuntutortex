% \chapter{26.11.2010}

\chapter{Harmonisch Eingeschwungener Zustand (HEZ)}

\begin{skills}
  \item Rechnen mit komplexen Zahlen
  \item Komplexe Wechselstromrechnung
  \item Additionstheoreme
\end{skills}


\paragraph{Voraussetzungen}, damit sich ein \acs{NWM} im \acs{HEZ} befindet:

\klausurhinweis{
  In der Klausur \emph{muss} bewiesen werden, dass sie erfüllt sind!
  Sonst gibt es massiv Punktabzug!
}

\begin{itemize}
  \item asymptotische Stabilität, d.\,h. alle natürlichen Frequenzen $\cn{A}_i$
    haben $\Re{\cn{A}_i} < 0$
  \item Das \acs{NWM} muss sich im eingeschwungenen Zustand befinden, \mbox{d.\,h.}\xspace
    $t-t_s$, $t-t_n >> \underset{1\leq i\leq n}{\operatorname{max}}
    \left(-\frac{1}{\Re{\cn{A}_i}}\right)$, die $-\frac{1}{\Re{\cn{A}_i}}$
    sind die Zeitkonstanten des \acs{NWM}s
  \item Für alle festen Quellen $e_k(t)$ gilt:
  
    $e_k(t)$ ist rein harmonisch für $t>t_k$, $k=1,\ldots,n_Q$, dabei ist
    $t_m = \underset{1\leq i\leq n_Q}{\operatorname{max}}(t_k)$,
    d.h. $t_m$ ist der Zeitpunkt, ab dem alle Quellen rein harmonisch sind!
\end{itemize}

% TODO: Zeichnung

\paragraph{Folgerungen:}
\begin{itemize}
  \item Alle Anfangswerte sind abgeklungen ($\rightarrow$EZ), $a_{zi}(t)=0$
  
  \item Alle Zweiggrößen sind rein harmonische Funktionen:\\
    $a(t)=\Re{\cn{A}\cdot e^{j\omega t}}
    \underset{\cn{A}=A\cdot e^{j\varphi_A}}{=}
    A\cos(\omega t+\varphi_A)$
    
    $\Rightarrow$ Das NWM lässt sich vollständig durch \acs{KWSR} beschreiben!
    
    \paragraph{Beispiel}
    
    \begin{align*}
      u_R(t) &= R\cdot i_R(t) 
      &&\underset{
        \begin{aligned}
          i_R(t)
            &= I_R\cdot\cos(\omega t+\varphi)  \\
            &= \Re{I_R\cdot e^{j\varphi}\cdot e^{j\omega t}}  \\
            &= \Re{\cn{I}_R\cdot e^{j\omega t}}
        \end{aligned}
      }{\xrightarrow{\hspace{1cm}}}
      &&\begin{aligned}
        u_R(t)
          &= \Re{R\cdot \cn{I}_R\cdot e^{j\omega t}} \\
          &= \Re{\cn{U}_R\cdot e^{j\omega t}}
      \end{aligned} \\
      %
      u_L(t) &= L\cdot \frac{\partial}{\partial t}i_L(t) 
      &&\underset{
        \begin{aligned}
          i_L(t) = \Re{\cn{I}_L\cdot e^{j\omega t}}
        \end{aligned}}
        {\xrightarrow{\hspace{1cm}}}
      &&\begin{aligned}
          u_L(t)
            &= \Re{L\cdot \frac{\partial}{\partial t} \cn{I}_L
              \cdot e^{j\omega t}}  \\
            &= \Re{j\omega L\cdot\cn{I}_L\cdot e^{j\omega t}} \\
            &= \Re{\cn{U}_L\cdot e^{j\omega t}}
        \end{aligned} \\
      %
      i_c(t) &= C\cdot \frac{\partial}{\partial t} u_C(t)
      &&\underset{
        \begin{aligned}
          u_C(t) = \Re{\cn{U}_C\cdot e^{j\omega t}}
        \end{aligned}}
        {\xrightarrow{\hspace{1cm}}}
      &&\begin{aligned}
          i_c(t)
            &= \Re{C\cdot \frac{\partial}{\partial t}\cn{U}_C
              \cdot e^{j\omega t}}  \\
            &= \Re{j\omega C\cdot \cn{U}_C\cdot e^{j\omega t}}
        \end{aligned}
    \end{align*}

    Das Einzige, was sich bei den \acs{ZGL} ändert, sind die Zeiger:
    \[\left.
      \begin{aligned}
        \cn{U}_R &= R\cdot \cn{I}_R  \\
        \cn{U}_L &= j\omega L\cdot \cn{I}_L  \\
        \cn{I}_C &= j\omega C\cdot \cn{U}_c
      \end{aligned}
      \right\}\text{ZGL der komplexen Wechselstromrechnung}\]
    (Für MGL/SGL geht es genauso!)
\end{itemize}

\clearpage
\section{Vorgehensweise HEZ}

\begin{enumerate}
%   \item Stelle $e(t)$ bzw. $e_\infty(t)$ auf in der Form:
%     \[e(t)=\sum_{k=1}^n E_k \cos(\omega_k t+\varphi_k)\qquad
%     \text{//$n$ ist hier \emph{nicht} die Ordnung des NWM!}\]
  \item Stelle $e(t)$ bzw. $e_\infty(t)$ auf in der Form:
    \[\colorbox{emphlight}{$\displaystyle e(t)=\sum_{k=1}^n E_k \cdot\cos(\omega_k t+\varphi_k)$}\]
    Dazu muss $e(t)$ ggf. mit Hilfe von Additionstheoremen umgeformt werden.
    Alle $E_k$, $\omega_k$, $\varphi_k$ sind zu definieren!\\
    Hilfreich: $V=V\cdot \cos(0\cdot t+0)$, falls $v(t) = V$

  \item Beweise, dass die Vorraussetzungen für HEZ erfüllt sind
    (Antwortsatz: Formelsammlung)

  \item Stelle $\cn H(j\omega)$ in Polarkoordinaten dar
    ($\cn H(j\omega)=\frac{\cn P(j\omega)}{\cn Q(j\omega)}$ mit $\cn P(j\omega)$,
    $\cn Q(j\omega)$ Polynome):
    \begin{align*}
      \cn{H}(j\omega)&\phantom{:}=:K(\omega)
        \cdot e^{j(\varphi_P(\omega)-\varphi_Q(\omega))}\\
      K(\omega) &:= |\cn{H}(j\omega)|
        = \sqrt{\frac{\Im{\cn{P}(j\omega)}^2 + \Re{\cn{P}(j\omega)}^2}
        {\Im{\cn{Q}(j\omega)}^2 + \Re{\cn{Q}(j\omega)}^2}}
        \qquad \text{// nicht vereinfachen!}\\
      \varphi_{P}(\omega)
        &:= \arctan\left({\frac{\Im{\cn P(j\omega)}}{\Re{\cn P(j\omega)}}}\right) +
        \begin{cases}
          0   &\text{falls} \Re{\cn P(j\omega)} > 0 \\
          \pi &\text{falls} \Re{\cn P(j\omega)} < 0
        \end{cases}\\
        % TODO
        % Muss jür \emph{jedes} $\omega_k$ einzeln geprüft werden, sodass
        % danach üfr jedes $\omega_k$ klar ist, zu welchem Fall es gehört
      \varphi_Q(\omega) &\phantom{:=} \text{analog zu } \varphi_P(\omega)
    \end{align*}
    Damit gilt (für \emph{eine} Quelle mit \emph{einer} Frequenz):
    \[\cn{A}=\cn{H}(j\omega)\cdot \cn{E}
      = E\cdot K(\omega)
        \cdot e^{j(\varphi_E+\varphi_P(\omega)-\varphi_Q(\omega))}\]
    Transformation in den Zeitbereich:
    \[\Rightarrow a(t) = \Re{\cn{A}\cdot e^{j\omega t}}
      = E \cdot K(\omega)
        \cdot \cos(\omega t+\varphi_E+\varphi_P(\omega)-\varphi_Q(\omega))\]


  \item Berechne $a(t)$ (näherungsweise) per Superposition im HEZ:
    \[a(t) \approx a_{\text{hez}}
      = \sum_{k=1}^{n}
        E_k\cdot K(\omega_k)\cdot\cos(
          \omega_k t+\varphi_{E,k}+\varphi_P(\omega_k)-\varphi_Q(\omega_k))\]
    $\varphi_P(\omega_k)$ und $\varphi_Q(\omega_k)$ müssen für jedes $k$
    bestimmt werden!
\end{enumerate}


%     Zuerst die Voraussetzungen beweisen, dann:
%     \[\cn{H}(j\omega) = \frac{\cn{A}}{\cn{E}}
%       = \frac{\cn{P}(j\omega)}{\cn{Q}(j\omega)}
%       \text{ in Polarkoordinaten darstellen: }
%       \cn{H}(j\omega):=K(\omega)
%         \cdot e^{j(\varphi_P(\omega)-\varphi_Q(\omega))} % TODO: check!
%       \]
%     D.h. Betrag ($K(\omega)$) und Argument ($\varphi_{P,Q}$) definieren:
%     \[\]
%     \[\varphi_{P,Q}(\omega)
%       := \arctan\left({\frac{\Im{}}{\Re{}}}\right) +
%         \begin{cases}
%           0   &\text{falls} \Re{} > 0 \\
%           \pi &\text{falls} \Re{} < 0
%         \end{cases}
%       \qquad \text{definieren!}
%     \]


\section{Komplexe Zahlen}

\begin{align*}
  \cn z = a+jb = r\cdot e^{j\varphi} = r\cdot (\cos\varphi + j\sin\varphi)
\end{align*}


\subsection{Betrag einer komplexen Zahl}\skilllabel{komplexbetrag}

\[|\cn z| = \sqrt{\Im{\cn z}^2+\Re{\cn z}^2}\]

\subsection{Winkel einer komplexen Zahl}\skilllabel{komplexwinkel}\hfill

\begin{minipage}{0.5\textwidth}
\begin{tikzpicture}[scale=0.8]
  \draw[->,>=latex] (-3,0) -- (3,0) node [below right] {$\operatorname{Re}$};
  \draw[->,>=latex] (0,-3) -- (0,3) node [left] {$\operatorname{Im}$};
  
  \coordinate[] (P) at (2,2) {};
  \coordinate[] (Ps) at (-2,-2) {};
  \draw[fill] (P) circle (0.05);
  \draw[fill] (Ps) circle (0.05);
  \draw (Ps) node[left] {$-\cn{z}$}
    -- (P) node [right] {$\cn{z} = x+jy\text{ mit }x,y>0$};
  \draw[blue, thick] (0,0) -| (P);
  \draw (1,0) arc (0:45:1) node [below=0.2] {$\varphi$};
\end{tikzpicture}
\end{minipage}
\hfill
\begin{minipage}{0.4\textwidth}
\[\color{blue}\left(\text{Hauptwert des $\arctan$: }
  \real\to\left(-\frac{\pi}{2};\frac{\pi}{2}\right)
  \right)\]
\end{minipage}

Für den Winkel gilt:
\[\angle \cn{z} = \arctan\left(\frac{y}{x}\right)
  \in \left(-\frac{\pi}{2};\frac{\pi}{2}\right) \]
für $-\cn{z}$ gilt: $\angle-\cn{z} = \angle z \pm\pi$,
gleichzeitig liefert der Hauptwert des $\arctan()$:
\[\arctan{\left(\frac{-y}{-x}\right) = \arctan\left(\frac{y}{x}\right)}
  =\angle\cn{z}\]
Der Hauptwert liefert Werte zischen $-\frac{\pi}{2}$ und $\frac{\pi}{2}$
zurück, er berechnet den Winkel von $\cn{z}$ korrekt, solange
\[\angle\cn{z} \in \left(-\frac{\pi}{2};\frac{\pi}{2}\right)
  \Leftrightarrow \cn{z} \text{ liegt in der rechten \acs{HE} }
  \Leftrightarrow \Re{\cn z} > 0\]
Andernfalls  muss noch die Korrektur "`$+\pi$"' dazuaddiert werden:
\[\colorbox{emphlight}{$\angle\cn{z} = \arctan\left(\frac{y}{x}\right)
  + \begin{cases}
      0 & \text{falls } \Re{\cn{z}}=x > 0 \text{ \quad// rechte \acs{HE}} \\
      \pi & \text{falls } \Re{\cn{z}}=x < 0 \text{ \quad// linke \acs{HE}}
    \end{cases}$}
\]



% \section{Erweiterung der Rechentechniken auf mehrere Quellen/eine Quelle
% mit mehreren Frequenzen}
% 
% \begin{itemize}
%   \item Forme $e(t)$ mit Hilfe von Additionstheoremen solange um, dass $e(t)$
%     folgende Form hat:
%     \[e(t) = \sum_{k=1}^{n} E_k\cdot \cos(\omega_k\cdot t+\varphi_k)
%       \qquad\begin{aligned}
%         &\text{// Hilfreich: } V=V\cdot \cos(0\cdot t+0)
%         \text{, falls } v(t) = V \\
%         &\text{// Alle } E_k, \omega_k, \varphi_k \text{ definieren!}
%       \end{aligned}\]
%   \item Per Superposition gilt:
%     \begin{align*} 
%       a_{hez}(t) &\;\;\;= \Re{
%         \sum_{k=1}^{n} \underbrace{\underbrace{
%           E_k\cdot e^{j\varphi_k}}_{\cn{E}_k}
%         \cdot \cn{H}(j\omega_k)}_{\cn{A}_k=\cn{H}(j\omega)\cdot\cn{E}_k}
%         \cdot e^{j\omega_k t}} \\
%       &\underset{\begin{matrix}\text{Polar-}\\\text{darst.}\end{matrix}}{=}\Re{
%         \sum_{k=1}^{n} E_k \cdot K(\omega_k)
%           \cdot e^{j(\omega_k\cdot t +\varphi_k
%             + \varphi_P(\omega_k) - \varphi_Q(\omega_k))}}  \\
%       &\;\;\;= \sum_{k=1}^{n}
%         E_k\cdot K(\omega_k)\cdot\cos(
%           \omega_k\cdot t+\varphi_k+\varphi_P(\omega_k)-\varphi_Q(\omega_k))
%     \end{align*}
%   \item \color{blue} Berechne für $e_k(t)$ den Anteil $a_{hez,k}(t)$ wie oben
%     für alle $k=1,\ldots,n_Q$.
%     
%     Dann gilt wieder per Superposition:
%     \[a_{hez}(t) = \sum_{k=1}^{n_Q} a_{hez,k}(t)\]
% \end{itemize}



% \chapter{03.12.2010}


% TODO: ESB für t > t_s
\section{Beispielaufgabe F05 A1 a),b)}

\begin{enumerate}[label= \bfseries\alph*)]
  \item 
    \begin{align*}
      \cn{H} &\;\;= \frac{\cn{I}_{C_2}}{\cn V}
        = j\omega C_2\cdot\frac{\cn{U}_{C_2}}{\cn V}  \\
        &\underset{\begin{matrix}
            \text{\footnotesize Spg.-}\\
            \text{\footnotesize Teiler}
          \end{matrix}}{=}
          j\omega C_2\cdot\frac{\frac{1}{j\omega C_2}\parallel(j\omega L+R)}
          {\frac{1}{j\omega C_2}\parallel(j\omega L+R)+\frac{1}{j\omega C_1}}\\
        &\;\;= \cancel{j\omega C_2}
          \cdot\frac{\cancel{\frac{1}{j\omega C_2}}\cdot (j\omega L+R)}
            {\frac{1}{j\omega C_2} \cdot (j\omega L+R) + \frac{1}{j\omega C_1}
              \cdot\left(j\omega L+R+\frac{1}{j\omega C_2}\right)}
          \cdot\frac{(j\omega)^2C_1C_2}
            {(j\omega)^2C_1C_2} \\
        &\;\;=\frac{(j\omega)^3C_1C_2L+(j\omega)^2C_1C_2R}
          {(j\omega)^2(C_1+C_2)L+(j\omega)\cdot R(C_1+C_2)+1}
    \end{align*}

  \item
    \begin{itemize}
      \item Stelle $e(t) = v(t)$ als Summe von $\cos$-Fkt. dar:
        \[v(t) = \underbrace{A\cdot \sin(\omega_0 t+\varphi_1)}_{
            \begin{aligned}
              E_1       &:= A \\
              \omega_1  &:= \omega_0  \\
              \Phi_1    &:= \varphi_1-\frac\pi2
            \end{aligned}}
          +\underbrace{B\cdot \cos(3\omega_0 t)}_{
            \begin{aligned}
              E_2       &:= B \\
              \omega_2  &:= 3\omega_0 \\
              \Phi_2    &:= 0
            \end{aligned}}
          =\sum_{k=1}^{2} E_k\cdot \cos(\omega_k t+\Phi_k)\]
      \item Voraussetzung für den HEZ erfüllt?
      
        \begin{itemize}[label=$\rightarrow$]
          \item Asympt. Stabilität!
          \item $t-t_s$, $t-t_m >> \underset{i}{\operatorname{max}}
            \left\{-\frac{1}{\Re{\cn{A}_i}}\right\}$
          \item Alle $e(t)$ für $t>t_m$ sind harmonisch!
        \end{itemize}
        
        \underline{$\Rightarrow$ Erfüllt.}
        
        Das als asympt. stabil angenommene \acs{NWM} befindet sich für\\
        $t-t_s>>\underset{i}{\operatorname{max}}
          \left\{-\frac{1}{\Re{\cn{A}_i}}\right\}$ näherungsweise im EZ.
          Da $v(t)$ rein harmonisch ist, sogar im \acs{HEZ}!
    \end{itemize}\medskip

  % TODO: Kasten unten rechts fehlt!
  \begin{minipage}{0.5\textwidth}
    $\cn{A}_i\colon$ Natürliche Frequenzen
    \[e^{\cn{A}_i\cdot t} = \underset{
        \begin{array}{c}
          \scriptstyle\text{klingt ab}\\
          \scriptstyle\text{für }t\to\infty,\\
          \scriptstyle\text{falls }\sigma<0
        \end{array}}{
      e^{\sigma\cdot t}} \cdot \underset{
        \begin{array}{c}
          \downarrow\\
          |e^{j\omega t}|=1
        \end{array}}
    {e^{j\omega t}}\]
  \end{minipage}
  \begin{minipage}{0.5\textwidth}
    \[\Rightarrow \frac{1}{\sigma}\mathrel{\widehat{=}} \text{Zeitkonstante}\]
    
    $t_m$: Ab diesem Zeitpunkt sind alle $e(t)$ harmonisch.
  \end{minipage}

\end{enumerate}


