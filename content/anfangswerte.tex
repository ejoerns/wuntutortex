% \chapter{19.11.2010}

\chapter{Anfangswerte}

\emph{Gibt es bei $t=t_s$ inkonsistente AW?}

$\to$ Das bedeutet, man kann AW vorgeben, die beim Schalten \emph{NICHT}
stetig übernommen werden.
Das bedeutet aber nur, dass es mind. einen AW-Vektor gibt, der beim Schalten
nicht stetig übernommen wird.

\paragraph{Beispiel:}\hfill\\

\begin{minipage}{0.5\textwidth}
\begin{circuitikz}
  \draw (1,0)
    to[short,o-] (0,0)
    -- (0,-1)
    to[V=$V_0$] (0,-4)
    -- (4,-4)
    to[C=$C$, v=$u_{C2}(t)$] (4,-2)
    to[C=$C$, v=$u_{C1}(t)$] (4,0)
    to[short,-o] (2,0);
  \draw (1,0) -- +(30:1);
\end{circuitikz}
\end{minipage}
\begin{minipage}{0.5\textwidth}
  \begin{itemize}
    \item Bei $u_{C_1}(t_s^-) = u_{C_2}(t_s^-) = 2V$, $V_0=5V$ ist die MGL
      für $t>t_s$ nicht erfüllt.\\
      $\Rightarrow$ $\underbrace{u_{C_1}(t_s^-)}_{\text{\parbox{1.75cm}{direkt \emph{vor} dem Schalten}}}
      \neq  \underbrace{u_{C_1}(t_s^+)}_{\text{\parbox{1.75cm}{direkt \emph{nach} dem Schalten}}}$
    \item Bei $u_{C_1}(t_s^-) = u_{C_2}(t_s^-) = 2,5V$, $V_0=5V$ ist die MGL
      für $t>t_s$ erfüllt.\\
      $\Rightarrow$ $u_{C_1}(t_s^-) = u_{C_1}(t_s^+) = 2,5V$\\
      Dies gilt aber eben \emph{NICHT} für \emph{alle} AW!
  \end{itemize}
\end{minipage}\medskip



\begin{description}
  \item[Vorgehensweise:]\hfill
    \begin{itemize}
      \item $n_A$, $n_R$ bestimmen für $t>t_s$
      \item zu beweisen: $n_R=0$ für $t>t_s$
        $\Rightarrow$ es gibt keine inkonsistenten AW bei $t=t_s$!\\
        ($n_R>0$: es gibt inkonsistente AW!)
        
      \begin{description}
        \item[Wiederholung]
          Gibt es keine gesteuerten Quellen im NWM, kann es nur 2 Typen von 
          zustandsreduzierenden Gleichungen geben:
          \begin{enumerate}
            \item MGL nur aus $C_s$ und/oder festen SPQ
            \item SGL nur aus $L_s$ und/oder festen STQ
          \end{enumerate}
      \end{description}
    \end{itemize}
\end{description}


\paragraph{Weiteres Beispiel}

[BEISPIEL]% TODO

\paragraph{Bsp zur diffbaren Variable}

[BEISPIEL]% TODO

\paragraph{Bei gesteuerten Quellen (wieder für $t>t_s$)}

\begin{itemize}
  \item $n_A$, $n_R$ bestimmen, wenn $n_R>0$: fertig!
  \item $\cn{H}(j\omega)$,
    $\cn{H}(\cn{s}) = \frac{\cn{P}(\cn{s})}{\cn{Q}(\cn{s})}$ aufstellen.\medskip
    
    $\cn{P}(\cn{s})$, $\cn{Q}(\cn{s})$ teilerfremd? Wenn nein: Polydiv!
  \item $n_A = \grad{(\cn{Q}(\cn{s}))}$? Wenn ja: fertig, da aus
    $n_A  = \grad{(\cn{Q}(\cn{s}))}$ folgt: $n=n_A$, $n_R=0$\\
    $\Rightarrow$ keine zustandsreduzierende Gleichung für $t>t_s$\\
    $\Rightarrow$ keine inkonsistenten AW bei $t=t_s$!
\end{itemize}


\klausurhinweis{
  Wenn sowohl Beweis für asymptotische Stabilität als auch
  für inkonsistente Anfangswerte verlangt wird:
  \begin{enumerate}
    \item Stabilität beweisen
    \item Mit den Ergebnissen argumentieren, ob es inkonsistente AW gibt oder nicht!
  \end{enumerate}
}

\section{Vorgehensweise Anfangswerte}

\begin{enumerate}
  \item Sind alle AW $x(t_s^+)$ gegeben
    (z.\,B. $u_C(t)$, $i_L(t)$, alle differenzierbare Variable
    für die ein AW vorgegeben werden kann)
  \item Sind alle AW für $t=t_s^-$ vorgegeben?
  \item Prüfe: ist das NWM für $t=t_s^-$ eingeschwungen, d.\,h.
    \begin{enumerate}
      \item beweise: Asymptotische Stabilität!
      \item Wartebedingung
    \end{enumerate}
  \item Das NWM ist eingeschwungen:
    \begin{itemize}
      \item Harmonisch eingeschwungener Zustand (HEZ)\\
        $\Rightarrow$ KWSR
      \item DC-Eingeschwungener Zustand (DC-EZ)\\
        $\Rightarrow$ DC-ESB
      \item Allgemein EZ\\
        $\Rightarrow$ Faltung\\
        Bsp.: Rechkeck-, Dreieck-Impuls,
    \end{itemize}
\end{enumerate}



